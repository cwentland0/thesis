\section{Reaction Models}

Two reaction models are used to simulate chemical reactions in this thesis: the laminar finite rate model and the flamelet/progress variable model. In both models, the $\reacIdx$th chemical reaction of a mechanism is described with the general form
%
\begin{equation}
	\specSumAll \stoichCoefSpecReac' \chi_{\specIdx} \xrightleftharpoons[]{} \specSumAll \stoichCoefSpecReac'' \chi_{\specIdx},
\end{equation}
%
where $\stoichCoefSpecReac'$ and $\stoichCoefSpecReac''$ are the stoichiometric coefficients of the $\specIdx$th species as reactants and products of the $\reacIdx$th reaction, respectively. The $\specIdx$th chemical species is denoted by the symbol $\chi_{\specIdx}$. A simple example of this format is the reaction of carbon monoxide and hydroperoxyl, or
%
\begin{equation}\label{eq:exampleReac}
	\carbonmonox + \hydroperoxyl \xrightleftharpoons[]{} \hydroxide + \carbondiox.
\end{equation}
%
The stoichiometry of this reaction is written in Table~\ref{tab:exampleReac}.

\begin{table}
	\centering
	\setlength{\tabcolsep}{12pt}
	\begin{tabular}{llll}
		\toprule
		$\specIdx$ & $\chi_{\specIdx}$ & $\stoichCoefSpec'$ & $\stoichCoefSpec''$ \\
		\midrule
		% 1 & $\methane$ & 1 & 0 \\
		% 2 & $\oxygen$ & 2 & 0 \\
		% 3 & $\carbondiox$ & 0 & 1 \\
		% 4 & $\water$ & 0 & 2 \\
		1 & $\carbonmonox$ & 1 & 0 \\
		2 & $\hydroperoxyl$ & 1 & 0 \\
		3 & $\hydroxide$ & 0 & 1 \\
		4 & $\carbondiox$ & 0 & 1 \\
		\bottomrule
	\end{tabular}
	\caption{\label{tab:exampleReac}Stoichiometry of example reaction, Eq.~\ref{eq:exampleReac}}
\end{table}

\subsection{Finite Rate Reactions}\label{sec:finiterate}

The laminar finite rate chemistry model is used to compute results for the one-dimensional model premixed flame in Chapter~\ref{chap:TransientFlame} and the nine-element combustor in Chapter~\ref{chap:NineElement}. In this model, the production rate of the $\specIdx$th species (in kg/m$^3$-s) is given by the relationship
%
\begin{equation}\label{eq:specProd}
	\prodRateSpec = \mwSpec \reacSumAll \left[\stoichCoefSpecReac'' - \stoichCoefSpecReac'\right] \rateofprogReac,
\end{equation}
%
where $\numReacs$ is the total number of reactions in the mechanism. The rate-of-progress of the $\reacIdx$th reaction is computed as
%
\begin{equation}\label{eq:rateOfProg}
	\rateofprogReac = \forwardRateReac \prod_{\specIdx=1}^{\numSpec} [\moleSpec]^{\stoichCoefSpecReac'} - \reverseRateReac \prod_{\specIdx=1}^{\numSpec} [\moleSpec]^{\stoichCoefSpecReac''},
\end{equation}
%
where $\forwardRateReac$ and $\reverseRateReac$ are the forward and reverse reaction rates, respectively. Here, $[\moleSpec]$ is the molar concentration of the $\specIdx$th species. The forward reaction rate (or the rate at which reactants are converted to products) is computed as an Arrhenius rate, given by the general form
%
\begin{equation}
	\forwardRateReac = A_\reacIdx \temperature^{b_\reacIdx} \text{exp} \left( \frac{-E_{a,\reacIdx}}{\gasConstUniv \temperature} \right),
\end{equation}
%
where $A_\reacIdx$ is the pre-exponential factor, $b_\reacIdx$ is the temperature exponent, and $E_{a,\reacIdx}$ is the activation energy of the $\reacIdx$th reaction. These constant factors are tabulated for each reaction in the mechanism, generally fit to match experimental results. Next, chemical reactions are assumed to progress at a much smaller time scale than that of transport phenomena, and thus any chemical reactions are assumed to be in local equilibrium. The reverse reaction rate can then be computed as
%
\begin{equation}
	\reverseRateReac = \frac{\forwardRateReac}{\equilConstReac},
\end{equation}
%
where $\equilConstReac$ is the equilibrium constant for the $\reacIdx$th reaction. The equilibrium constant is computed by
%
\begin{equation}
	\equilConstReac = \text{exp}\left(-\specSumAll \left[\stoichCoefSpecReac'' - \stoichCoefSpecReac'\right] \gibbsSpec\right) \left[\frac{101,325 \; \text{Pa}}{\temperature \gasConstUniv}\right]^{\specSumAll \left[\stoichCoefSpecReac'' - \stoichCoefSpecReac'\right]},
\end{equation}
%
where $\gibbsSpec$ is the Gibbs free energy of the $\specIdx$th species in the mixture, given by
%
\begin{equation}\label{eq:gibbsSpec}
	\gibbsSpec = \frac{\enthSpec}{\gasConstSpec \temperature} - \frac{\entropySpec}{\gasConstSpec}.
\end{equation}
%
After the species production rates (Eq.~\ref{eq:specProd}) are computed, they are substituted into the corresponding scalar transport equations for each chemical species, as outlined in the sixth equation in Eq.~\ref{eq:gemsGovVecs}.

For an \textit{irreversible} reaction, it is assumed that the reaction only proceeds in the forward direction, i.e. $\reverseRateReac = 0$. Irreversible reactions are denoted by a single rightward arrow, such as
%
\begin{equation}
	\methyl + \oAtom \rightarrow \hAtom + \hydrogen + \carbonmonox.
\end{equation}
%
While no reaction is truly irreversible, in some cases the reverse reaction is so unlikely (relative to the forward reaction) that it may be safely ignored. This greatly simplifies the calculation of the reaction rate-of-progress (Eq.~\ref{eq:rateOfProg}). The one-dimensional model premixed flame in Chapter~\ref{chap:TransientFlame} utilizes a single irreversible reaction, and some reactions of the mechanism used for the nine-element combustor in Chapter~\ref{chap:NineElement} are treated as irreversible.

Note that the reaction mechanism used in simulations of the nine-element combustor studied in Chapter~\ref{chap:NineElement} includes reactions involving third-body effects in the low-pressure limit. These corrections are either of the Lindemann--Hinshelwood form~\cite{Hinshelwood1926} or Troe form~\cite{Gilbert1983}.

\subsection{Flamelet/Progress Variable Model}\label{sec:fpv}
%
In order to capture complex phenomena in turbulent combustion (such as local extinction and ignition), particularly for complex hydrocarbon fuels, chemical mechanisms for finite rate reaction models often account for dozens of species and hundreds of reactions. Further, as mentioned previously, the characteristic spatio-temporal scales of turbulent flames are extremely small. The cost of accurately modeling these processes during CFD calculations can be extremely computationally expensive.

Laminar flamelet modeling, originally introduced by Peters~\cite{Peters1984}, attempts to simplify chemical transport and reaction modeling for turbulent non-premixed flames by treating the flame as an ensemble of localized flamelets. This operates by introducing the fuel mixture fraction,
%
\begin{equation}\label{eq:mixFracDef}
    \mixFrac \defEq \frac{ \nu \mf_{\text{f}} - \mf_{\text{ox}} + \mf^0_{\text{ox}} }{ \nu \mf^0_{\text{f}} + \mf^0_{\text{ox}} },
\end{equation}
%
where $\mf_{\text{f}}$ and $\mf_{\text{ox}}$ are the local fuel and oxidizer mass fractions, respectively, $\mf^0_{\text{f}}$ is the mass fraction of fuel in the fuel stream, and $\mf^0_{\text{ox}}$ is the mass fraction of oxidizer in the oxidizer stream. The stoichiometric mass ratio is defined as
%
\begin{equation}
	\stoichCoef \defEq \frac{\stoichCoef'_{\text{ox}} \mw_{\text{ox}}}{\stoichCoef'_{\text{f}} \mw_{\text{f}}}.
\end{equation}
%
The mixture fraction acts as an independent coordinate that varies from 0 (in the pure oxidizer stream) to 1 (in the pure fuel stream) normal to the flame surface defined by $\mixFrac = \mixFrac_{\text{st}}$, the stoichiometric mixture fraction (i.e. where $\nu \mf_{\text{f}} = \mf_{\text{ox}}$). The transport equation for the fuel mass fraction is given as the seventh equation in Eq.~\ref{eq:gemsGovVecs}. The species mass fraction fields are related to the fuel mixture fraction field by the steady flamelet equations
%
\begin{equation}\label{eq:steadyFlamelet}
	-\density \scalarDiss \ode{^2 \mfSpec}{\mixFrac^2} = \prodRateSpec,
\end{equation}
%
where the scalar dissipation rate, in s$^{-1}$, is given by
\begin{equation}
	\scalarDiss = 2 \massDiffMixFrac \left[\nabla \mixFrac \cdot \nabla \mixFrac\right].
\end{equation}
%
In the original flamelet model formulation, a library of solutions to Eq.~\ref{eq:steadyFlamelet} is precomputed from one-dimensional counterflow diffusion flame simulations (e.g., via FlameMaster~\cite{flamemaster}). This provides a unique mapping from the local fuel mixture fraction and dissipation rate to individual mass fractions, i.e. $\mfSpec = \mfSpec(\mixFrac, \; \scalarDiss)$. As a result, the mixture fraction transport equation need only be accounted for, replacing the species transport equations.

The original flamelet model, however, does not explicitly model any chemical reaction effects. In order to account for this, an additional tracking variable, the \textit{progress variable} $\progVar$ was introduced by Charles Pierce~\cite{Pierce2001}. This takes the form
%
\begin{equation}\label{eq:progVarDef}
	\progVar \defEq \sum_{\specIdxTwo = 1}^{\numProg} \mf_{\specIdx_{\specIdxTwo}},
\end{equation}
%
where $\numProg$ is the total number of species mass fractions which make up the progress variable. In general, the constituent mass fractions are chosen to be those of reaction products or late-stage intermediates, as they are an indicator of reaction progress and provide a unique mapping to all chemical states (in combination with the mixture fraction). For example, the simulations of the truncated single-element combustor presented in Section~\ref{sec:cvrc} use carbon dioxide, carbon monoxide, and diatomic hydrogen. The precomputed flamelet library thus provides unique mappings,
%
\begin{equation}
	\mfSpec = \mfSpec(\mixFrac, \; \progVar), \quad \prodRateFPV = \prodRateFPV(\mixFrac, \; \progVar).
\end{equation}
%
The transport equation for the progress variable is given as the eighth equation in Eq.~\ref{eq:gemsGovVecs}. In summary, the FPV model introduces two scalar transport equations, and precomputes a library of steady flamelet solutions which provide a unique mapping from these two scalars to the species mass fraction fields and progress variable rate of production. Especially when large, complex reaction mechanisms are used, this greatly reduces the computational cost of simulating non-premixed combusting flow systems.

Note that, traditionally, the FPV model also provides unique mappings for the temperature, density, and transport/thermodynamic quantities as part of the flamelet tabulation (e.g., $\temperature = \temperature(\mixFrac, \; \progVar)$, $\dynVisc = \dynVisc(\mixFrac, \; \progVar)$). GEMS does not include these mappings, and instead solves the energy conservation equation directly. While this approach ensures conservation of energy, it also incurs significant computational costs due to the additional conservation equation and the direct calculation of transport and thermodynamic properties. Note that when a filter is applied for LES, an additional transport equation must be solved for the mean mixture fraction variance ($\widetilde{\mixFrac''^2}$). The mean mixture fraction variance is also used in the flamelet library mapping (i.e. $\mfSpec = \mfSpec(\widetilde{\mixFrac}, \; \widetilde{\mixFrac''^2}, \; \widetilde{\progVar})$). Recall that all results in this thesis do not compute any explicit filtering, hence the mean mixture fraction variance is not accounted for here.