

The first practical reacting flow case of this thesis is now detailed, namely a model of the continuously-variable resonance combustor (CVRC) with a truncated combustion chamber. The CVRC experiment originated in work by Yu \textit{et al.}~\cite{Yu2008,Yu2009,Yu2012} at Purdue University, whereby a single-element, gaseous propellant, coaxial dump injection rocket combustor was designed to allow for continuous actuation of the oxidizer injection post. This actuation revealed that certain oxidizer post lengths gave rise to high-frequency combustion instabilities. The CVRC has become a useful numerical benchmark case for simulating rocket combustion~\cite{Garby2013,Nguyen2018} and investigating the mechanics which contribute to combustion instabilities in rocket engines~\cite{HarvazinskiCVRCOrig,Harvazinski2016,HarvazinskiCVRCflamelet}.

In this work, the full injector and combustor geometry of the CVRC is not modeled, unlike in~\cite{HarvazinskiCVRCOrig}. Instead, as will be detailed in the following section, the oxidizer injection choke manifold is neglected and the combustion chamber is truncated well downstream of the dump plane. This has the effect of decreasing computational costs, while still including much of the complex combusting flow phenomena such the mixing shear layer, recirculation of hot products, and large-scale transport and mixing in the combustion chamber. Acoustic forcing is not applied at the outlet to mimic the combustion instability, as this would necessitate complex boundary treatments in any PROMs, which are beyond the scope of this work. The reader is directed to work by Huang \textit{et al.}~\cite{Huang2022a} for a thorough investigation of artificial boundary forcing treatments for PROMs.

\subsection{Full-order Model}

The truncated CVRC case presented here generally follows that investigated by Harvazinski and Shimizu~\cite{HarvazinskiCVRCflamelet}, with a truncated combustion chamber. The geometry is displayed in Fig.~\ref{fig:cvrcGeom}. The oxidizer post extends approximately 14 cm upstream of the dump plane ($x = $ 0 m), with a diameter of 2.05 cm. The annular fuel injection port has an outer diameter of 23.09 cm and an inner diameter of 22.27 cm, extends 30.55 cm upstream of the dump plane, and enters the oxidizer stream 10.18 cm upstream of the dump plane. The combustion chamber has a diameter of 4.5 cm, and extends approximately 14 cm downstream of the dump plane. The oxidizer inlet injects 42.35\% gaseous oxygen and 57.65\% water vapor by mass at 1,030 K, with a specified mass flow rate of 0.32 kg/s. Gaseous methane is injected through the fuel duct at 300 K, with a mass flow rate of 0.027 kg/s. The computational mesh is composed of 2,637,771 hexahedral cells, resulting in a total number of degrees of freedom $\numDOF =$ 18,464,397. No-slip, adiabatic conditions are enforced at the domain walls, and a subsonic characteristic boundary condition is applied at the outlet.

\begin{figure}
	\centering
	\includegraphics[width=0.8\linewidth]{Chapters/HPROMResults/Images/cvrc/geom_hrProbe.png}
    \caption{\label{fig:cvrcGeom}
	Truncated CVRC geometry with $x-z$ plane slice at $\timeVar = 5.5$ ms.}
\end{figure}

Combustion is modeled using the FPV approach~\cite{Pierce2001} detailed in Section~\ref{sec:fpv}. As detailed previously, a library of steady diffusion flame solutions is pre-computed off-line, generating a lookup table mapping from the mixture fraction $Z$ and the progress variable $C$ to individual species mass fractions, i.e., $Y_i = Y_i(Z,C)$. For this case, the steady flame solutions are solved using the FlameMaster software~\cite{flamemaster} with the GRI-Mech 1.2 methane combustion mechanism~\cite{griMech}. The GRI-Mech 1.2 mechanism contains 32 chemical species and 177 reactions. All gases are treated as thermally perfect gases, and thermodynamic quantities (specific heats, enthalpy, and entropy) as well as transport properties (mass diffusivity, viscosity, and thermal conductivity) are computed from polynomial functions of temperature developed by McBride et al.~\cite{McBride1993} and described in Section~\ref{subsec:gasModels}.

\begin{figure}
	\begin{minipage}{0.99\linewidth}
		\raisebox{-0.5\height}{\includegraphics[width=0.84\linewidth,trim={0.5em 0.1em 0.5em 0.1em},clip]{Chapters/HPROMResults/Images/cvrc/example_pressure_z.png}}
		\raisebox{-0.5\height}{\includegraphics[width=0.14\linewidth,trim={0.0em 0.1em 0.0em 0.1em},clip]{Chapters/HPROMResults/Images/cvrc/example_pressure_x.png}}
	\end{minipage}
	\begin{minipage}{0.99\linewidth}
		\raisebox{-0.5\height}{\includegraphics[width=0.84\linewidth,trim={0.5em 0.1em 0.5em 0.1em},clip]{Chapters/HPROMResults/Images/cvrc/example_temperature_z.png}}
		\raisebox{-0.5\height}{\includegraphics[width=0.14\linewidth,trim={0.0em 0.1em 0.0em 0.1em},clip]{Chapters/HPROMResults/Images/cvrc/example_temperature_x.png}}
	\end{minipage}
	\begin{minipage}{0.99\linewidth}
		\raisebox{-0.5\height}{\includegraphics[width=0.84\linewidth,trim={0.5em 0.1em 0.5em 0.1em},clip]{Chapters/HPROMResults/Images/cvrc/example_xVel_z.png}}
		\raisebox{-0.5\height}{\includegraphics[width=0.14\linewidth,trim={0.0em 0.1em 0.0em 0.1em},clip]{Chapters/HPROMResults/Images/cvrc/example_xVel_x.png}}
	\end{minipage}
	\begin{minipage}{0.99\linewidth}
		\raisebox{-0.5\height}{\includegraphics[width=0.84\linewidth,trim={0.5em 0.1em 0.5em 0.1em},clip]{Chapters/HPROMResults/Images/cvrc/example_mixFrac_z.png}}
		\raisebox{-0.5\height}{\includegraphics[width=0.14\linewidth,trim={0.0em 0.1em 0.0em 0.1em},clip]{Chapters/HPROMResults/Images/cvrc/example_mixFrac_x.png}}
	\end{minipage}
	\caption{\label{fig:cvrcFOMSlices}From top to bottom: pressure, temperature, axial velocity, and fuel mixture fraction slices at $\timeVar = 5.5$ ms.}
\end{figure}

The FOM simulation is computed with a physical time step of $\dtFOM = 0.1 \; \mu$s. The fluid domain is initialized with a pressure of 1 MPa, and the combustion chamber is filled with hot products at 2,000 K to initiate combustion. Initial transients exit the domain before data collection begins at 5 ms of simulation time. Data snapshots are collected over a 0.5 ms window, sampled every five time steps, corresponding to roughly one flow-through period in the combustion chamber. This results in 1,001 snapshots (including the initial condition at $\timeVar =$ 5 ms) of the conservative and primitive states. Representative snapshot slices of various flow fields are displayed in Fig.~\ref{fig:cvrcFOMSlices}. Pressure point monitor measurements are collected from the dump plane corner at approximately $x = \left(0, \; 0, \; 2.25\right)$ cm, and unsteady heat release point monitor measurements are collected from the reacting mixing layer at approximately $x = \left(5, \; 0, \; 1.5\right)$ cm.

\begin{figure}
	\centering
	\includegraphics[width=0.8\linewidth]{Chapters/HPROMResults/Images/cvrc/cvrc_pod_energy_0p5ms.png}
	\caption{\label{fig:cvrcPODEnergy}POD residual energy decay for truncated CVRC conservative and primitive state datasets.}
\end{figure}

\begin{figure}
	\begin{minipage}{0.48\linewidth}
		\includegraphics[width=0.99\linewidth,trim={0.5em 0.5em 0.5em 0.5em},clip]{Chapters/HPROMResults/Images/cvrc/projection_error_primitive.png}
		\caption{\label{fig:cvrcProjErrPrim}Primitive variables time-average projection error.}
	\end{minipage} \hspace{0.5em}
	\begin{minipage}{0.48\linewidth}
		\includegraphics[width=0.99\linewidth,trim={0.5em 0.5em 0.5em 0.5em},clip]{Chapters/HPROMResults/Images/cvrc/projection_error_conservative.png}
		\caption{\label{fig:cvrcProjErrCons}Conservative variables time-average projection error.}
	\end{minipage}
\end{figure}

Figure~\ref{fig:cvrcPODEnergy} displays the POD residual energy for the model rocket combustor data. Achieving 1\%, 0.1\%, and 0.01\% residual energy for the conservative dataset requires 51, 123, and 245 basis modes, respectively. For the primitive dataset, these levels require 44, 134, and 267 modes respectively. This decay is significantly slower than that observed for 2D open cavity flow (Fig.~\ref{fig:cavityPODEnergy}), and is indicative of the difficulty with which linear trial spaces capture the highly non-linear flow physics which characterize rocket combustion. This is further emphasized in the projection error plots shown in Figs.~\ref{fig:cvrcProjErrPrim} and~\ref{fig:cvrcProjErrCons}, where it is apparent that the fuel mixture fraction and progress variable fields induce much higher error levels at a given trial basis dimension than the primary flow fields.

As with the 2D cavity case, this full-order model should not be considered a well-resolved or accurate representation of the complex combustion phenomena which occur in real liquid-propellant rocket combustor. Although this model is derived from numerical experiments~\cite{HarvazinskiCVRCOrig} which exhibited reasonable predictions of certain quantities of interest, the results presented here should be contextualized strictly as approximating the system dynamics modeled by the underlying numerical solver, which this thesis does not claim to perfectly represent reality.

\subsection{Unsampled PROMs}

The performance of the unsampled PROMs are again examined before proceeding to HPROMs. Discussion is restricted to MP-LSVT PROMs, and the exclusion of Galerkin and LSPG PROMs is explained at the end of this section. A trial basis dimension and time step study is again conducted. The trial basis dimension $\numPrimModes$ is again evaluated at 25-mode intervals from 25 modes to 200 modes. Four time step sizes are examined: $\Delta \timeVar \in \{ 0.1, \; 0.25, \; 0.5, \; 1 \} \; \mu \text{s}$, or 1, 2.5, 5, and 10 times that of the FOM simulation.

\begin{figure}
	\centering
	\includegraphics[width=0.7\linewidth]{Chapters/HPROMResults/Images/cvrc/unsampled/unsampled_avg_mode_Average_errorRaw.png}
	\caption{\label{fig:cvrcUnsampledROMErrVsModes}CVRC unsampled PROM time-average error, various $\dt$.}
\end{figure}

The time-average error results are displayed in Fig.~\ref{fig:cvrcUnsampledROMErrVsModes}. Note that, in general, the error levels induced for this case are significantly higher than those observed in the cavity flow case. Whereas the unsampled cavity MP-LSVT PROMs easily achieved less than 1\% relative $\ell^2$ error for all time steps for $\numPrimModes \ge 50$, this is only achieved here for $\dt = \dtFOM$, $\numPrimModes = 200$. This is yet another indicator of the difficulty in accurately modeling reacting flows. However, observe the similar trend that moderate increases in the PROM time step result in negligible increase in PROM error, while larger time steps greatly diminish the PROM's accuracy and quickly saturates accuracy improvements with trial basis enrichment. For $\dt = 10 \times \dtFOM$, unlike the cavity whose time-average error saturated at approximately 0.8\%, error saturates at roughly 4\% for this case.

\begin{figure}
	\begin{minipage}{0.49\linewidth}
		\includegraphics[width=0.99\linewidth]{Chapters/HPROMResults/Images/cvrc/unsampled/pressure_probe_unsampled_modes.png}
		\subcaption{\label{fig:cvrcUnsampledROMProbesPress}Dump plane pressure.}
	\end{minipage}
	\begin{minipage}{0.49\linewidth}
		\includegraphics[width=0.99\linewidth]{Chapters/HPROMResults/Images/cvrc/unsampled/heat_probe_unsampled_modes.png}
		\subcaption{\label{fig:cvrcUnsampledROMProbesHeat}Shear layer heat release.}
	\end{minipage}
	\caption{\label{fig:cvrcUnsampledROMProbes}CVRC probe measurements, $\dt = 5 \times \dtFOM$, various $\numPrimModes$}
\end{figure}

The effect of trial basis enrichment is visualized in Fig.~\ref{fig:cvrcUnsampledROMProbes}. Interestingly, the pressure signal at the dump plane corner, shown in Fig.~\ref{fig:cvrcUnsampledROMProbesPress}, is relatively easy to capture, although significant smearing of small scale fluctuations can be observed for $\numPrimModes = 25$, and less so for $\numPrimModes = 50$. For a trial basis dimension of $\numPrimModes = 100$, even very small scale fluctuations are captured well. Comparisons for the reacting mixing layer heat release, however, are more telling. Figure~\ref{fig:cvrcUnsampledROMProbesHeat} indicates that $\numPrimModes = 25$ results in dramatic over- and under-prediction of heat release across the entire simulation period. Increasing the trial basis resolution to $\numPrimModes = 50$ improves this error, but does not eliminate it. Again, $\numPrimModes = 100$ results in excellent time-accurate reconstruction of the unsteady heat release in the reacting mixing layer.

\begin{figure}
	\begin{minipage}{0.49\linewidth}
		\includegraphics[width=0.99\linewidth]{Chapters/HPROMResults/Images/cvrc/unsampled/pressure_probe_unsampled_lspg.png}
		\subcaption{\label{fig:cvrcLSPGProbe}Probe measurements.}
	\end{minipage}
	\begin{minipage}{0.49\linewidth}
		\includegraphics[width=0.99\linewidth]{Chapters/HPROMResults/Images/cvrc/unsampled/condition_number.png}
		\subcaption{\label{fig:cvrcLSPGCond}Condition number, $\numConsModes,\numPrimModes = 100$}
	\end{minipage}
	\caption{CVRC unsampled MP-LSVT and LSPG PROM comparisons.}
\end{figure}

The absence of any analysis for LSPG PROMs above and throughout the remained of this section is now explained. As documented by Huang \textit{et al.} for a 2D single-element rocket combustor~\cite{Huang2022}, Galerkin and LSPG PROMs exhibit increased stiffness in the resulting linear temporal evolution system, compared to that generated by the equivalent MP-LSVT PROM. The result is drastically degraded stability and accuracy, such that all investigated Galerkin PROMs were unstable and LSPG PROMs exhibited more than double the error measured for MP-LSVT PROMs with the same trial basis dimension. For the truncated CVRC case investigated here, \textit{all} LSPG PROMs computed unstable solutions, even with significant basis enrichment. Figure~\ref{fig:cvrcLSPGProbe} indicates the effect of this instability, with a rapid explosion in pressure measurements at roughly $\timeVar = 5.025$ ms for $\numConsModes \in \{100, \; 200\}$. The MP-LSVT PROM with $\numPrimModes = 100$ is both stable and noticeably more accurate than the equivalent LSPG PROM. Investigating further, Fig.~\ref{fig:cvrcLSPGCond} displays the evolution of the condition number of the first LSPG and MP-LSVT Newton iteration, i.e. $\kappa \left(\left[\testBasis^{\newtonIdx}\right]^\top \testBasis^{\newtonIdx}\right)$. This roughly measures the stiffness of the PROM linear solve, and reveals that while both methods experience similar ill conditioning at first, the conditioning of the MP-LSVT PROM gradually lessens while that of the LSPG PROM gradually increases and eventually explodes.

\subsection{Mesh Sampling and Load Balancing}

The pre-processing stage of developing hyper-reduced PROMs for the truncated CVRC is now discussed. Although a brief overview of hyper-reduction computational processes was given in Section~\ref{sec:cavity}, more attention is given here to some of the nuances related to load balancing and MPI communications for parallel computations.

\begin{table}
	\centering
	\begin{tabular}{ lllllll }
	\toprule
	Sampling Rate (\%) & 0.025 & 0.0375 & 0.05 & 0.075 & 0.1 & 0.175 \\
	\midrule
	Cores & 2 & 2 & 2 & 2 & 3 & 5 \\
	Cells/core (approx.) & 330 & 495 & 660 & 989 & 880 & 923 \\
	\bottomrule
	\toprule
	Sampling Rate (\%) & 0.25 & 0.375 & 0.5 & 0.75 & 1 &  \\
	\midrule
	Cores & 7 & 10 & 13 & 20 & 26 & \\
	Cells/core (approx.) & 942 & 989 & 1,015 & 989 & 1,015 & \\
	\bottomrule
	\end{tabular}
	\caption{\label{tab:cvrcSampProcs}Partitioning for CVRC HPROM sample meshes.}
\end{table}

As with the cavity flow case, the four sampling algorithms described in Section~\ref{subsec:sampAlgos} are studied, and analyses are conducted for a range of sampling rates and gappy POD regressor basis dimensions. The sampling rates investigated and the parallel partitioning for each sample mesh (regardless of the sampling algorithm used) are summarized in Table~\ref{tab:cvrcSampProcs}, and are again adjusted to ensure that the load balance is approximately equal to 1,000 cells per core.

\begin{figure}
	\begin{minipage}{0.99\linewidth}
		\raisebox{-0.5\height}{\includegraphics[width=0.84\linewidth,trim={0.5em 0.5em 0.5em 0.5em},clip]{Chapters/HPROMResults/Images/cvrc/deim/iblank/random_iblank_z.png}}
		\raisebox{-0.5\height}{\includegraphics[width=0.14\linewidth,trim={0.0em 0.1em 0.0em 0.1em},clip]{Chapters/HPROMResults/Images/cvrc/deim/iblank/random_iblank_x.png}}
	\end{minipage}
	\begin{minipage}{0.99\linewidth}
		\raisebox{-0.5\height}{\includegraphics[width=0.84\linewidth,trim={0.5em 0.5em 0.5em 0.5em},clip]{Chapters/HPROMResults/Images/cvrc/deim/iblank/eigenvec_iblank_z.png}}
		\raisebox{-0.5\height}{\includegraphics[width=0.14\linewidth,trim={0.0em 0.1em 0.0em 0.1em},clip]{Chapters/HPROMResults/Images/cvrc/deim/iblank/eigenvec_iblank_x.png}}
	\end{minipage}
	\begin{minipage}{0.99\linewidth}
		\raisebox{-0.5\height}{\includegraphics[width=0.84\linewidth,trim={0.5em 0.5em 0.5em 0.5em},clip]{Chapters/HPROMResults/Images/cvrc/deim/iblank/greedy_carlberg_iblank_z.png}}
		\raisebox{-0.5\height}{\includegraphics[width=0.14\linewidth,trim={0.0em 0.1em 0.0em 0.1em},clip]{Chapters/HPROMResults/Images/cvrc/deim/iblank/greedy_carlberg_iblank_x.png}}
	\end{minipage}
	\begin{minipage}{0.99\linewidth}
		\raisebox{-0.5\height}{\includegraphics[width=0.84\linewidth,trim={0.5em 0.5em 0.5em 0.5em},clip]{Chapters/HPROMResults/Images/cvrc/deim/iblank/greedy_ben_iblank_z.png}}
		\raisebox{-0.5\height}{\includegraphics[width=0.14\linewidth,trim={0.0em 0.1em 0.0em 0.1em},clip]{Chapters/HPROMResults/Images/cvrc/deim/iblank/greedy_ben_iblank_x.png}}
	\end{minipage}
	\caption{\label{fig:cvrcIBlankSlices}Example sample meshes for $\numSamps = 0.25\% \times \numDOF$, $\numResModes = 300$ for various sampling algorithms. From top to bottom: random, eigenvector-based, GNAT V1, and GNAT V2.}
\end{figure}

Visually inspecting several indicative samples meshes, with slices in Fig.~\ref{fig:cvrcIBlankSlices} and iso-surfaces in Fig.~\ref{fig:cvrcIBlankIso}, it is clear that the spatial distribution of the sample mesh points varies drastically between sampling strategies. Random sampling, as expected, selects points in a random pattern, including a significant number of points in the oxidizer duct. This might be considered computationally wasteful, as the physics in the oxidizer duct are quite simple, maintaining a roughly fixed temperature, velocity, and chemical composition, with only minor variations in pressure. Eigenvector-based sampling and both variants of GNAT sampling, on the other hand, display a vastly different result. In each case, sample mesh elements are clustered around the reacting shear layer downstream of the dump plane. This is particularly clear for eigenvector-based sampling, in which the sampling points form a conical shape aligned with the mixing layer of the annular fuel injection stream. This result is not entirely surprising, as the physics in the shear layer are by far the most complex in the system, featuring sharp gradients and strong non-linearities. Interestingly, both GNAT sampling methods generate extremely tight clusters in the vicinity of the fuel dump region, although the second variant selects some element in the downstream combustion chamber region. These findings are largely similar to those for the cavity flow case: among the greedy sampling algorithms, eigenvector sampling generates a larger, more diffuse sample mesh, while GNAT sampling clusters sampling points closely. 

\begin{figure}
	\centering
	\begin{minipage}{0.45\linewidth}
		\includegraphics[width=0.99\linewidth,trim={0.5em 0.5em 0.5em 0.5em},clip]{Chapters/HPROMResults/Images/cvrc/deim/iblank/random_iblank_iso.png}
		\subcaption{Random}
	\end{minipage}
	\begin{minipage}{0.45\linewidth}
		\includegraphics[width=0.99\linewidth,trim={0.5em 0.5em 0.5em 0.5em},clip]{Chapters/HPROMResults/Images/cvrc/deim/iblank/eigenvec_iblank_iso.png}
		\subcaption{Eigenvector}
	\end{minipage}

	\centering
	\begin{minipage}{0.45\linewidth}
		\includegraphics[width=0.99\linewidth,trim={0.5em 0.5em 0.5em 0.5em},clip]{Chapters/HPROMResults/Images/cvrc/deim/iblank/greedy_carlberg_iblank_iso.png}
		\subcaption{GNAT V1}
	\end{minipage}
	\begin{minipage}{0.45\linewidth}
		\includegraphics[width=0.99\linewidth,trim={0.5em 0.5em 0.5em 0.5em},clip]{Chapters/HPROMResults/Images/cvrc/deim/iblank/greedy_ben_iblank_iso.png}
		\subcaption{GNAT V2}
	\end{minipage}
	\caption{\label{fig:cvrcIBlankIso}Directly-sampled cell isosurfaces, $\numSamps = 0.25\% \times \numDOF$, $\numResModes = 300$, various sampling algorithms.}
\end{figure}

How these differences in sampling affect the load balancing of the mesh partitions is quantified in Fig.~\ref{fig:cvrcMeshStats}. Figure~\ref{fig:cvrcCellPerPart}
displays the average number of cells per partition in the sample mesh as a percentage of the full mesh, when the sample mesh is split into ten partitions. Unsurprisingly, random sampling generates the largest meshes for all sampling rates, as the disperse sampling requires a larger number of auxiliary cells. Conversely, both GNAT sampling algorithms produce much smaller, compact sample meshes due to the tight clustering of sample points and commensurately fewer auxiliary cells they require. At low sampling rates, eigenvector-based sampling produces similar sample mesh sizes to random sampling, but approaches sizes similar to GNAT sampling at higher sampling rates. 

\begin{figure}
	\begin{minipage}{0.49\linewidth}
		\includegraphics[width=0.99\linewidth]{Chapters/HPROMResults/Images/cvrc/deim/stats/cvrc_partition_stats.png}
		\subcaption{\label{fig:cvrcCellPerPart}Average cells per partition, \% of total mesh.}
	\end{minipage}
	\begin{minipage}{0.49\linewidth}
		\includegraphics[width=0.99\linewidth]{Chapters/HPROMResults/Images/cvrc/deim/stats/cvrc_partition_comms.png}
		\subcaption{\label{fig:cvrcMPIComm}Total point-to-point MPI communications.}
	\end{minipage}
	\caption{\label{fig:cvrcMeshStats}Sample mesh partition statistics, 10 partitions, various sampling rates.}
\end{figure}

For much of the same reasons, the opposite is true for the volume of inter-core communications that are required for a given sample mesh. Figure~\ref{fig:cvrcMPIComm} shows that in many cases, random sampling produces sample meshes which require no point-to-point MPI communications whatsoever. This is a function of the aforementioned disperse sampling which results in a large number of disjoint graphs which can be effectively separated by the partitioning software. This is not the case at extremely low sampling rates, which are dominated by mesh elements selected by the QDEIM procedure. Both GNAT sampling algorithms produce nearly identical volumes of communication at all sampling rates. This follows logically from the fact that tight sampling clusters require edge cuts to ensure effective load balancing. Eigenvector-based sampling induces a lower volume than GNAT sampling at lower sampling rates, but more at higher sampling rates. 

\begin{figure}
	\begin{minipage}{0.49\linewidth}
		\includegraphics[width=0.99\linewidth]{Chapters/HPROMResults/Images/cvrc/deim/samp_timing_wrt_samprate.png}
		\subcaption{\label{fig:cvrcSampCostSamp}W/r/t sampling rate, $\numResModes = 300$.}
	\end{minipage}
	\begin{minipage}{0.49\linewidth}
		\includegraphics[width=0.99\linewidth]{Chapters/HPROMResults/Images/cvrc/deim/samp_timing_wrt_modes.png}
		\subcaption{\label{fig:cvrcSampCostModes}W/r/t $\numResModes$, $\numSamps = 0.25\% \times \numDOF$.}
	\end{minipage}
	\caption{\label{fig:cvrcSampCost}Sample selection cost, relative to cost of a single FOM iteration.}
\end{figure}

Finally, Fig.~\ref{fig:cvrcSampCost} examines the offline cost of evaluating the various sampling algorithms. The cost is non-dimensionalized by the core-hours required to compute one FOM iteration. Of course, the cost of random sampling is nearly zero, scales negligibly with respect to the sampling rate~\ref{fig:cvrcSampCostSamp}, and is not influenced by the number of modes retained in the regression basis $\deimBasis$ (Fig.~\ref{fig:cvrcSampCostModes}). The greedy algorithms, on the other hand, incur noticeable computational cost. On the whole, the original GNAT sampling algorithm is significantly less expensive, obviously because it requires only $\numResModes$ evaluations of the greedy metric, while Peherstorfer's variant and eigenvector-based sampling require approximately $\numSamps / \numVars$ evaluations. Eigenvector-based sampling is only slightly more costly than Peherstorfer's GNAT variant. All greedy methods appear to scale sublinearly with respect to $\numResModes$, while eigenvector-based sampling and Peherstorfer's GNAT variant seem to scale exponentially with respect to the sampling rate. Although these latter two methods cost over ten FOM iterations to compute for sampling rates above $0.1\%$, recall that this accounts to 0.2\% of the total FOM cost (5,000 iterations). 

\subsection{Hyper-reduced PROMs}

The same analysis conducted for the MP-LSVT PROMs of the 2D cavity case are repeated for the truncated CVRC. All HPROMs utilize a trial basis dimension of $\numPrimModes = 100$, as this was shown above to accurately reconstruct the pointwise unsteady heat release. Again, all gappy POD regressor bases are generated from the POD of the conservative field dataset. The sampling rates, number of partitions, and approximate number of cells per partition are given in Table~\ref{tab:cvrcSampProcs}.

% \begin{figure}
% 	\begin{minipage}{0.46\linewidth}
% 		\includegraphics[width=0.99\linewidth]{Chapters/HPROMResults/Images/cvrc/deim/err_contour_random_dt2p5e-7.png}
% 		\subcaption{Random}
% 	\end{minipage}
% 	\begin{minipage}{0.53\linewidth}
% 		\includegraphics[width=0.99\linewidth]{Chapters/HPROMResults/Images/cvrc/deim/err_contour_eigenvec_dt2p5e-7.png}
% 		\subcaption{Eigenvector}
% 	\end{minipage}

% 	\begin{minipage}{0.46\linewidth}
% 		\includegraphics[width=0.99\linewidth]{Chapters/HPROMResults/Images/cvrc/deim/err_contour_gnat1_dt2p5e-7.png}
% 		\subcaption{GNAT, V1}
% 	\end{minipage}
% 	\begin{minipage}{0.53\linewidth}
% 		\includegraphics[width=0.99\linewidth]{Chapters/HPROMResults/Images/cvrc/deim/err_contour_gnat2_dt2p5e-7.png}
% 		\subcaption{GNAT, V2}
% 	\end{minipage}
% 	\caption{CVRC HPROM time-average error contours with respect to gappy POD regressor dimension and sampling rate, $\dt = 2.5 \times \dtFOM$, various sampling algorithms}
% \end{figure}

\begin{figure}
	\begin{minipage}{0.46\linewidth}
		\includegraphics[width=0.99\linewidth]{Chapters/HPROMResults/Images/cvrc/deim/err_contour_random_dt5e-7.png}
		\subcaption{Random}
	\end{minipage}
	\begin{minipage}{0.53\linewidth}
		\includegraphics[width=0.99\linewidth]{Chapters/HPROMResults/Images/cvrc/deim/err_contour_eigenvec_dt5e-7.png}
		\subcaption{Eigenvector}
	\end{minipage}

	\begin{minipage}{0.46\linewidth}
		\includegraphics[width=0.99\linewidth]{Chapters/HPROMResults/Images/cvrc/deim/err_contour_gnat1_dt5e-7.png}
		\subcaption{GNAT, V1}
	\end{minipage}
	\begin{minipage}{0.53\linewidth}
		\includegraphics[width=0.99\linewidth]{Chapters/HPROMResults/Images/cvrc/deim/err_contour_gnat2_dt5e-7.png}
		\subcaption{GNAT, V2}
	\end{minipage}
	\caption{\label{fig:cvrcSampledROMErrContourDt5e-7}CVRC HPROM time-average error contours with respect to gappy POD regressor dimension and sampling rate, $\dt = 5 \times \dtFOM$, various sampling algorithms}
\end{figure}

Time-average $\ell^2$ error contour plots for $\dt \in \{5, \; 10\} \times \dtFOM$, for various $\numResModes$ and $\numSamps$, are shown in Figs.~\ref{fig:cvrcSampledROMErrContourDt5e-7} and~\ref{fig:cvrcSampledROMErrContourDt1e-6}. In this case, all HPROMs remain stable, though those simulations which accrued over 20\% error (indicated by bright yellow) can be considered complete failures. For random and eigenvector-based sampling, the natural tendency for accuracy to improve with increased $\numSamps$ and $\numResModes$ is observed again. Interestingly, both variants of GNAT sampling seem to perform better at higher sampling rates for lower $\numResModes$, though these algorithms tend to perform very poorly across most configurations. Again, eigenvector-based sampling performs remarkably well, generating stable HPROMs with less than 5\% time-average error for $\numResModes \ge 250$. Random sampling also appears capably of generating similar levels of accuracy, albeit at much higher sampling rates.

\begin{figure}
	\begin{minipage}{0.46\linewidth}
		\includegraphics[width=0.99\linewidth]{Chapters/HPROMResults/Images/cvrc/deim/err_contour_random_dt1e-6.png}
		\subcaption{Random}
	\end{minipage}
	\begin{minipage}{0.53\linewidth}
		\includegraphics[width=0.99\linewidth]{Chapters/HPROMResults/Images/cvrc/deim/err_contour_eigenvec_dt1e-6.png}
		\subcaption{Eigenvector}
	\end{minipage}

	\begin{minipage}{0.46\linewidth}
		\includegraphics[width=0.99\linewidth]{Chapters/HPROMResults/Images/cvrc/deim/err_contour_gnat1_dt1e-6.png}
		\subcaption{GNAT, V1}
	\end{minipage}
	\begin{minipage}{0.53\linewidth}
		\includegraphics[width=0.99\linewidth]{Chapters/HPROMResults/Images/cvrc/deim/err_contour_gnat2_dt1e-6.png}
		\subcaption{GNAT, V2}
	\end{minipage}
	\caption{\label{fig:cvrcSampledROMErrContourDt1e-6}CVRC HPROM time-average error contours with respect to gappy POD regressor dimension and sampling rate, $\dt = 10 \times \dtFOM$, various sampling algorithms}
\end{figure}

Figure~\ref{fig:cvrcSampledROMErrVsTime} displays the cost-accuracy tradeoff incurred by varying the sampling rate, for fixed $\numResModes = 300$. As observed previously, decreasing the sampling rate tends to degrade simulation accuracy, but enables commensurate computational cost savings. Eigenvector-based sampling is able to generate HPROMs which achieve three to four orders of magnitude speedup over the FOM with minimal loss of accuracy with respect to the unsampled PROM. At extremely low sampling rates, though, the time-average error grows to over 8-10\%. Random sampling is capable of achieving similar speedup results, though this tends to incur unacceptable degredation of accuracy, particularly at very low sampling rates. In general, both GNAT sampling variants are unable to produce accurate HPROMs for any sampling rate investigated, measuring over 10\% error in almost all cases.

\begin{figure}
	% \begin{minipage}{0.49\linewidth}
	% 	\includegraphics[width=0.99\linewidth]{Chapters/HPROMResults/Images/cvrc/deim/sampled_dt2p5e-7_Average_errorRaw_pareto.png}
	% 	\subcaption{$\dt = 2.5 \times \dtFOM$}
	% \end{minipage}
	\begin{minipage}{0.49\linewidth}
		\includegraphics[width=0.99\linewidth]{Chapters/HPROMResults/Images/cvrc/deim/sampled_dt5e-7_Average_errorRaw_pareto.png}
		\subcaption{\label{fig:cvrcSampledROMErrVsTimeDt5e-7}$\dt = 5 \times \dtFOM$}
	\end{minipage}
	% \centering
	\begin{minipage}{0.49\linewidth}
		\includegraphics[width=0.99\linewidth]{Chapters/HPROMResults/Images/cvrc/deim/sampled_dt1e-6_Average_errorRaw_pareto.png}
		\subcaption{$\dt = 10 \times \dtFOM$}
	\end{minipage}
	\caption{\label{fig:cvrcSampledROMErrVsTime}CVRC HPROM time-average error vs. CPU-time speedup, various $\dt$}
\end{figure}

Figure~\ref{fig:cvrcSampledROMProbes} illustrates how the sampling algorithms and sampling rates affect the long-term reconstruction of the unsteady pressure signal at the dump plane corner (marked in Fig.~\ref{fig:cvrcGeom}). The average error for Fig.~\ref{fig:cvrcSampledROMProbe0p075} and~\ref{fig:cvrcSampledROMProbe0p25} are marked in Fig.~\ref{fig:cvrcSampledROMErrVsTimeDt5e-7} by an ``X'' and diamond, respectively. Almost all of the displayed simulations remain reasonably accurate up to $t = 5.1$ ms. As indicated by the above discussion, however, both GNAT variants appear incapable of generating accurate reconstructions beyond this point, either exploding or significantly deviating from the FOM pressure signal. For $\numSamps = 0.25 \times \numDOF$, random sampling is able to maintain an accurate reconstruction for $\timeVar < 5.35$ ms, but deteriorates near the end of the simulation. This is not the case for $\numSamps = 0.075 \times \numDOF$, where random sampling explodes shortly after $\timeVar = 5.25$ ms. On the other hand, eigenvector-based sampling is able to faithfully reconstruct the pressure signal in both instances, though for $\numSamps = 0.075 \times \numDOF$ the signal begins to deteriorate noticeably near the end of the simulation period.

\begin{figure}
	\begin{minipage}{0.49\linewidth}
		\includegraphics[width=0.99\linewidth]{Chapters/HPROMResults/Images/cvrc/deim/pressure_probe_deim_dt5e-7_samp0p00075.png}
		\subcaption{\label{fig:cvrcSampledROMProbe0p075}$\numSamps = 0.075\% \times \numDOF$}
	\end{minipage}
	\begin{minipage}{0.49\linewidth}
		\includegraphics[width=0.99\linewidth]{Chapters/HPROMResults/Images/cvrc/deim/pressure_probe_deim_dt5e-7_samp0p0025.png}
		\subcaption{\label{fig:cvrcSampledROMProbe0p25}$\numSamps = 0.25\% \times \numDOF$}
	\end{minipage}
	\caption{\label{fig:cvrcSampledROMProbes}CVRC MP-LSVT HPROM probe measurements, $\numPrimModes = 100$, $\numResModes = 300$, $\dt = 5 \times \dtFOM$.}
\end{figure}

These observations are emphasized further by field plots drawn from the end of the simulation period, $\timeVar = 5.5$ ms, for $\numSamps = 0.25\%$. Figure~\ref{fig:cvrcDEIMPressSlices} shows pressure slices, where it is abundantly clear that the HPROMs produced by random sampling and the GNAT variants have devolved into high-amplitude oscillations. Although the GNAT variants appear to avoid the extremely high-frequency oscillations observed in the random sampling cases, the point is ultimately irrelevant due to overall deviation from the solution. Eigenvector-based sampling computes a remarkably similar pressure field, though some of the very high-frequency content is absent in the combustion chamber (though this may be due entirely to an under-resolved trial space). 

\begin{figure}
	\begin{minipage}{0.99\linewidth}
		\raisebox{-0.5\height}{\includegraphics[width=0.84\linewidth,trim={0.5em 0.5em 0.5em 0.5em},clip]{Chapters/HPROMResults/Images/cvrc/example_pressure_z.png}}
		\raisebox{-0.5\height}{\includegraphics[width=0.14\linewidth,trim={0.0em 0.1em 0.0em 0.1em},clip]{Chapters/HPROMResults/Images/cvrc/example_pressure_x.png}}
	\end{minipage}
	\begin{minipage}{0.99\linewidth}
		\raisebox{-0.5\height}{\includegraphics[width=0.84\linewidth,trim={0.5em 0.5em 0.5em 0.5em},clip]{Chapters/HPROMResults/Images/cvrc/deim/contours/random/random_pressure_z.png}}
		\raisebox{-0.5\height}{\includegraphics[width=0.14\linewidth,trim={0.0em 0.1em 0.0em 0.1em},clip]{Chapters/HPROMResults/Images/cvrc/deim/contours/random/random_pressure_x.png}}
	\end{minipage}
	\begin{minipage}{0.99\linewidth}
		\raisebox{-0.5\height}{\includegraphics[width=0.84\linewidth,trim={0.5em 0.5em 0.5em 0.5em},clip]{Chapters/HPROMResults/Images/cvrc/deim/contours/eigenvec/eigenvec_pressure_z.png}}
		\raisebox{-0.5\height}{\includegraphics[width=0.14\linewidth,trim={0.0em 0.1em 0.0em 0.1em},clip]{Chapters/HPROMResults/Images/cvrc/deim/contours/eigenvec/eigenvec_pressure_x.png}}
	\end{minipage}
	\begin{minipage}{0.99\linewidth}
		\raisebox{-0.5\height}{\includegraphics[width=0.84\linewidth,trim={0.5em 0.5em 0.5em 0.5em},clip]{Chapters/HPROMResults/Images/cvrc/deim/contours/greedy_carlberg/greedy_carlberg_pressure_z.png}}
		\raisebox{-0.5\height}{\includegraphics[width=0.14\linewidth,trim={0.0em 0.1em 0.0em 0.1em},clip]{Chapters/HPROMResults/Images/cvrc/deim/contours/greedy_carlberg/greedy_carlberg_pressure_x.png}}
	\end{minipage}
	\begin{minipage}{0.99\linewidth}
		\raisebox{-0.5\height}{\includegraphics[width=0.84\linewidth,trim={0.5em 0.5em 0.5em 0.5em},clip]{Chapters/HPROMResults/Images/cvrc/deim/contours/greedy_ben/greedy_ben_pressure_z.png}}
		\raisebox{-0.5\height}{\includegraphics[width=0.14\linewidth,trim={0.0em 0.1em 0.0em 0.1em},clip]{Chapters/HPROMResults/Images/cvrc/deim/contours/greedy_ben/greedy_ben_pressure_x.png}}
	\end{minipage}
	\caption{\label{fig:cvrcDEIMPressSlices}CVRC HPROM pressure slices, $\timeVar = 5.5$ ms, $\numSamps$ = 0.25\%, $\numResModes = 300$, $\dt = 5 \times \dtFOM$. From top to bottom: FOM, random, eigenvector, GNAT V1, and GNAT V2 sampling.}
\end{figure}

Similar results are observed in the temperature field, shown in Fig.~\ref{fig:cvrcDEIMTempSlices}. The result produced by eigenvector-based sampling captures the mixing and advection of the reactant shear layer, though there is again noticeable smearing and spurious oscillations near the flame front. Much like with the pressure field, random sampling produces an HPROM which devolves into high-frequency oscillations in the mixing layer, while both GNAT variants produce large oscillations throughout the combustion chamber. 

\begin{figure}
	\begin{minipage}{0.99\linewidth}
		\raisebox{-0.5\height}{\includegraphics[width=0.84\linewidth,trim={0.5em 0.5em 0.5em 0.5em},clip]{Chapters/HPROMResults/Images/cvrc/example_temperature_z.png}}
		\raisebox{-0.5\height}{\includegraphics[width=0.14\linewidth,trim={0.0em 0.1em 0.0em 0.1em},clip]{Chapters/HPROMResults/Images/cvrc/example_temperature_x.png}}
	\end{minipage}
	\begin{minipage}{0.99\linewidth}
		\raisebox{-0.5\height}{\includegraphics[width=0.84\linewidth,trim={0.5em 0.5em 0.5em 0.5em},clip]{Chapters/HPROMResults/Images/cvrc/deim/contours/random/random_temperature_z.png}}
		\raisebox{-0.5\height}{\includegraphics[width=0.14\linewidth,trim={0.0em 0.1em 0.0em 0.1em},clip]{Chapters/HPROMResults/Images/cvrc/deim/contours/random/random_temperature_x.png}}
	\end{minipage}
	\begin{minipage}{0.99\linewidth}
		\raisebox{-0.5\height}{\includegraphics[width=0.84\linewidth,trim={0.5em 0.5em 0.5em 0.5em},clip]{Chapters/HPROMResults/Images/cvrc/deim/contours/eigenvec/eigenvec_temperature_z.png}}
		\raisebox{-0.5\height}{\includegraphics[width=0.14\linewidth,trim={0.0em 0.1em 0.0em 0.1em},clip]{Chapters/HPROMResults/Images/cvrc/deim/contours/eigenvec/eigenvec_temperature_x.png}}
	\end{minipage}
	\begin{minipage}{0.99\linewidth}
		\raisebox{-0.5\height}{\includegraphics[width=0.84\linewidth,trim={0.5em 0.5em 0.5em 0.5em},clip]{Chapters/HPROMResults/Images/cvrc/deim/contours/greedy_carlberg/greedy_carlberg_temperature_z.png}}
		\raisebox{-0.5\height}{\includegraphics[width=0.14\linewidth,trim={0.0em 0.1em 0.0em 0.1em},clip]{Chapters/HPROMResults/Images/cvrc/deim/contours/greedy_carlberg/greedy_carlberg_temperature_x.png}}
	\end{minipage}
	\begin{minipage}{0.99\linewidth}
		\raisebox{-0.5\height}{\includegraphics[width=0.84\linewidth,trim={0.5em 0.5em 0.5em 0.5em},clip]{Chapters/HPROMResults/Images/cvrc/deim/contours/greedy_ben/greedy_ben_temperature_z.png}}
		\raisebox{-0.5\height}{\includegraphics[width=0.14\linewidth,trim={0.0em 0.1em 0.0em 0.1em},clip]{Chapters/HPROMResults/Images/cvrc/deim/contours/greedy_ben/greedy_ben_temperature_x.png}}
	\end{minipage}
	\caption{\label{fig:cvrcDEIMTempSlices}CVRC HPROM temperature slices, $\timeVar = 5.5$ ms, $\numSamps$ = 0.25\%, $\numResModes = 300$, $\dt = 5 \times \dtFOM$. From top to bottom: FOM, random, eigenvector, GNAT V1, and GNAT V2 sampling.}
\end{figure}

\subsection{Effects of Sampling Criteria}\label{subsec:cvrcDOFSamp}

Three approaches for evaluating the greedy sampling algorithm metric and selecting those degrees of freedom to be appended to the sample set were described in Section~\ref{subsec:samplingCriteria}: agnostic, post-sampling, and comprehensive sampling. The comprehensive method has been claimed to provide a more complete description of the sampling metric for coupled fluid flow systems, and is used for all hyper-reduction results presented in this thesis. In this section, the validity of this claim is investigated. The performance of HPROMs generated by the post-sampling and comprehensive techniques are thus compared. Random sampling via post-sampling is omitted, as it is not a greedy sampling algorithm. The same sample mesh sizes as those listed in Table~\ref{tab:cvrcSampProcs} are again used for all post-sampling meshes. 

\begin{figure}
	\begin{minipage}{0.99\linewidth}
		\raisebox{-0.5\height}{\includegraphics[width=0.84\linewidth,trim={0.5em 0.5em 0.5em 0.5em},clip]{Chapters/HPROMResults/Images/cvrc/dof_samp/iblank/eigenvec_iblank_z.png}}
		\raisebox{-0.5\height}{\includegraphics[width=0.14\linewidth,trim={0.0em 0.1em 0.0em 0.1em},clip]{Chapters/HPROMResults/Images/cvrc/dof_samp/iblank/eigenvec_iblank_x.png}}
	\end{minipage}
	\begin{minipage}{0.99\linewidth}
		\raisebox{-0.5\height}{\includegraphics[width=0.84\linewidth,trim={0.5em 0.5em 0.5em 0.5em},clip]{Chapters/HPROMResults/Images/cvrc/dof_samp/iblank/greedy_carlberg_iblank_z.png}}
		\raisebox{-0.5\height}{\includegraphics[width=0.14\linewidth,trim={0.0em 0.1em 0.0em 0.1em},clip]{Chapters/HPROMResults/Images/cvrc/dof_samp/iblank/greedy_carlberg_iblank_x.png}}
	\end{minipage}
	\begin{minipage}{0.99\linewidth}
		\raisebox{-0.5\height}{\includegraphics[width=0.84\linewidth,trim={0.5em 0.5em 0.5em 0.5em},clip]{Chapters/HPROMResults/Images/cvrc/dof_samp/iblank/greedy_ben_iblank_z.png}}
		\raisebox{-0.5\height}{\includegraphics[width=0.14\linewidth,trim={0.0em 0.1em 0.0em 0.1em},clip]{Chapters/HPROMResults/Images/cvrc/dof_samp/iblank/greedy_ben_iblank_x.png}}
	\end{minipage}
	\caption{\label{fig:cvrcIBlankSlicesDOF}Example sample meshes for $\numSamps = 0.25\% \times \numDOF$, $\numResModes = 300$ for various sampling algorithms, using the post-sampling approach. From top to bottom: random, eigenvector-based, GNAT V1, and GNAT V2.}
\end{figure}

Figure~\ref{fig:cvrcIBlankSlicesDOF} provide indicative sample mesh slices for post-sampling meshes, reflecting the same mesh size and regression basis dimension $\numResModes$ as those displayed in Fig.~\ref{fig:cvrcIBlankSlices}. The two approaches generate similar meshes, and close inspection is required to notice differences. One might notice slightly tighter clustering of sampled points for eigenvector-based sampling, while the GNAT sample meshes are slightly more diffuse.

\begin{figure}
	\begin{minipage}{0.46\linewidth}
		\includegraphics[width=0.99\linewidth]{Chapters/HPROMResults/Images/cvrc/dof_samp/err_contour_random_dof_dt5e-7.png}
		\subcaption{Random}
	\end{minipage}
	\begin{minipage}{0.53\linewidth}
		\includegraphics[width=0.99\linewidth]{Chapters/HPROMResults/Images/cvrc/dof_samp/err_contour_eigenvec_dof_dt5e-7.png}
		\subcaption{Eigenvector}
	\end{minipage}

	\begin{minipage}{0.46\linewidth}
		\includegraphics[width=0.99\linewidth]{Chapters/HPROMResults/Images/cvrc/dof_samp/err_contour_gnat1_dof_dt5e-7.png}
		\subcaption{GNAT, V1}
	\end{minipage}
	\begin{minipage}{0.53\linewidth}
		\includegraphics[width=0.99\linewidth]{Chapters/HPROMResults/Images/cvrc/dof_samp/err_contour_gnat2_dof_dt5e-7.png}
		\subcaption{GNAT, V2}
	\end{minipage}
	\caption{\label{fig:cvrcSampledROMErrContourDOFDt5e-7}CVRC HPROM time-average error contours for sample meshes constructed via the post-sampling approach, with respect to gappy POD regressor dimension and sampling rate, $\dt = 5 \times \dtFOM$, various sampling algorithms}
\end{figure}

The resulting simulation accuracy with respect to the sample rate $\numSamps$ and the regression basis dimension $\numResModes$ also remains largely the same, as seen in Fig.~\ref{fig:cvrcSampledROMErrContourDOFDt5e-7}. Eigenvector-based sampling performs quite well at low sampling rates, random sampling performs well at high sampling rates, and both GNAT algorithms tend to perform poorly at any sampling rate. The general trend of improved accuracy with increased $\numResModes$ is observed again, most noticeably with eigenvector-based sampling.

\begin{figure}
	\begin{minipage}{0.49\linewidth}
		\includegraphics[width=0.99\linewidth]{Chapters/HPROMResults/Images/cvrc/dof_samp/sampled_dt5e-7_Average_errorRaw_pareto.png}
		\caption{\label{fig:cvrcSampledROMErrVsTimeDOF}CVRC HPROM time-average error vs. CPU-time speedup, $\dt = 5 \times \dtFOM$}
	\end{minipage}
	\begin{minipage}{0.49\linewidth}
		\includegraphics[width=0.99\linewidth]{Chapters/HPROMResults/Images/cvrc/dof_samp/pressure_probe_deim_dof_dt5e-7_samp0p0025.png}
		\caption{\label{fig:cvrcSampledROMProbesDOF}CVRC HPROM probe sampled via post-sampling, $\numPrimModes = 100$, $\numSamps = 0.25\% \times \numDOF$, $\numResModes = 300$, $\dt = 5 \times \dtFOM$.}
	\end{minipage}
\end{figure}

Examining the cost-accuracy tradeoff in Fig.~\ref{fig:cvrcSampledROMErrVsTimeDOF} emphasizes that the post-sampling approach has little, if any, influence on the accuracy and computational cost savings of the HPROMs. Except at very high sampling rates, where post-sampling appears to \textit{improve} the accuracy of the GNAT algorithms, the lines representing post-sampling results lie nearly on top of those representing comprehensive sampling. Pressure probe results in Fig.~\ref{fig:cvrcSampledROMProbesDOF} display nearly identical performance to those shown in Fig.~\ref{fig:cvrcSampledROMProbe0p25}.

The results shown above indicate that the choice of sampling criterion plays a minimal role in enabling fast and accurate HPROMs, relative to the extreme sensitivity to the sampling algorithm. This logically follows from the idea that certain variables make outsized contributions to the greedy sampling metric, e.g. the temperature vs. pressure field, and so both approaches result in similar sample meshes despite non-dimensionalization of the data prior to generation of the regression basis $\deimBasis$. In either case, those mesh elements which incur the greatest error are still selected.

\subsection{Effects of Dual Basis Gappy POD}\label{subsec:cvrcDualBasis}

All results shown thus far have used a single gappy POD regression basis, $\deimBasis \inRTwo{\numDOF}{\numResModes}$. Following the justification outlined in Section~\ref{subsec:stateApproxDEIM}, the regression basis has been chosen such that $\deimBasis = \consTrialSpace$; that is, the regression basis is computed from snapshots of the conservative fields. This approach has worked well, but alternative approaches have suggested representing components of the fully-discrete residual with separate gappy POD approximations (as in brief notes in~\cite{Tezaur2017} and~\cite{Grimberg2021}). In Section~\ref{subsec:dualBasis}, a dual-basis formulation was proposed by which the fully-discrete residual is separated into the time integrator $\consVecDt$ and all spatial discretization terms $\rhsFunc{\consVec, \; \timeVar}$, with corresponding gappy POD regression bases $\deimBasisCons \inRTwo{\numDOF}{\numConsModes}$ and $\deimBasisRhs \inRTwo{\numDOF}{\numRHSModes}$ respectively. For the sake of completeness, HPROMs constructed via this approach are investigated below. Only one set of basis sizes is investigated: $\numResModes = \numConsModes = 300$ and $\numRHSModes = 800$. Unlike in the above Section~\ref{subsec:cvrcDOFSamp}

\begin{figure}
	\begin{minipage}{0.99\linewidth}
		\raisebox{-0.5\height}{\includegraphics[width=0.84\linewidth,trim={0.5em 0.5em 0.5em 0.5em},clip]{Chapters/HPROMResults/Images/cvrc/dualModes/iblank/eigenvec_iblank_z.png}}
		\raisebox{-0.5\height}{\includegraphics[width=0.14\linewidth,trim={0.0em 0.1em 0.0em 0.1em},clip]{Chapters/HPROMResults/Images/cvrc/dualModes/iblank/eigenvec_iblank_x.png}}
	\end{minipage}
	\begin{minipage}{0.99\linewidth}
		\raisebox{-0.5\height}{\includegraphics[width=0.84\linewidth,trim={0.5em 0.5em 0.5em 0.5em},clip]{Chapters/HPROMResults/Images/cvrc/dualModes/iblank/greedy_carlberg_iblank_z.png}}
		\raisebox{-0.5\height}{\includegraphics[width=0.14\linewidth,trim={0.0em 0.1em 0.0em 0.1em},clip]{Chapters/HPROMResults/Images/cvrc/dualModes/iblank/greedy_carlberg_iblank_x.png}}
	\end{minipage}
	\begin{minipage}{0.99\linewidth}
		\raisebox{-0.5\height}{\includegraphics[width=0.84\linewidth,trim={0.5em 0.5em 0.5em 0.5em},clip]{Chapters/HPROMResults/Images/cvrc/dualModes/iblank/greedy_ben_iblank_z.png}}
		\raisebox{-0.5\height}{\includegraphics[width=0.14\linewidth,trim={0.0em 0.1em 0.0em 0.1em},clip]{Chapters/HPROMResults/Images/cvrc/dualModes/iblank/greedy_ben_iblank_x.png}}
	\end{minipage}
	\caption{\label{fig:cvrcIBlankSlicesDual}Example sample meshes for $\numSamps = 0.25\% \times \numDOF$, $\numResModes = 300$ for various sampling algorithms, using the post-sampling approach. From top to bottom: random, eigenvector-based, GNAT V1, and GNAT V2.}
\end{figure}

The sample meshes generated by the dual-basis sampling procedure are virtually identical to those generated by the single-basis approach, as seen in Fig.~\ref{fig:cvrcIBlankSlicesDual}. Close inspection reveals minor differences, but follow the same trends as those seen in Fig.~\ref{fig:cvrcIBlankSlices}. Despite the marked similarities of the sample meshes, the performance of the online HPROMs are strikingly different, as shown in Fig.~\ref{fig:cvrcSampledROMErrVsTimeDual}. For a given sample rate $\numSamps$, the dual-basis formulation halves the computational cost savings and nearly doubles the error (or even becomes unstable). Although the relative discrepancies between the sampling algorithms persist (eigenvector-based sampling performing the best, followed by random sampling, and GNAT tending to fail outright).

\begin{figure}
	\centering
	\includegraphics[width=0.6\linewidth]{Chapters/HPROMResults/Images/cvrc/dualModes/sampled_dt5e-7_Average_errorRaw_pareto.png}
	\caption{\label{fig:cvrcSampledROMErrVsTimeDual}CVRC HPROM time-average error vs. CPU-time speedup, $\dt = 5 \times \dtFOM$}
\end{figure}

\begin{figure}
	\begin{minipage}{0.49\linewidth}
		\includegraphics[width=0.99\linewidth]{Chapters/HPROMResults/Images/cvrc/dualModes/pressure_probe_deim_dual_random_eig_dt5e-7_samp0p0025.png}
	\end{minipage}
	\begin{minipage}{0.49\linewidth}
		\includegraphics[width=0.99\linewidth]{Chapters/HPROMResults/Images/cvrc/dualModes/pressure_probe_deim_dual_gnat_dt5e-7_samp0p0025.png}
	\end{minipage}
	\caption{\label{fig:cvrcSampledROMProbesDual}CVRC HPROM probe sampled via post-sampling, $\numPrimModes = 100$, $\numSamps = 0.25\% \times \numDOF$, $\numResModes = 300$, $\dt = 5 \times \dtFOM$.}
\end{figure}

Pressure monitor results in Fig.~\ref{fig:cvrcSampledROMProbesDual} only emphasize the comparatively poor performance of the dual-basis formulation. The accuracy of the random and eigenvector-based HPROMs deteriorates much more rapidly (around $\timeVar = 5.2$ ms), and even the GNAT HPROMs are wholly unstable (opposed to greatly inaccurate). These results indicate that the dual-basis formulation does not benefit the performance of the HPROMs despite a supposed improvement in the accuracy of the constituent gappy POD regressions. This is, in some sense, a comfort, inasmuch as the simplest solution (a single gappy POD basis) appears to be the best. Of course, increased granularity of the fully-discrete residual (e.g. generating separate approximations for the inviscid and viscous fluxes) might improve on this, but the introduces an exceptional level of complexity in the Gauss--Newton normal equations and was not deemed a worthwhile investigation. 