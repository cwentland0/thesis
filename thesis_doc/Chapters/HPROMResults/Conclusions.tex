\section{Conclusions}

In this chapter, scalable hyper-reduced PROMs were tested for several multi-scale and multi-physics systems of increasing dimension and complexity. Beginning with a non-reacting transonic open cavity flow case, residual-based PROM methods including LSPG and MP-LSVT were shown to be superior to the standard Galerkin projection method. The MP-LSVT method further exhibited excellent accuracy in modeling extremely stiff rocket combustor systems, including single-element and linear nine-element combustors, for which the LSPG method was incapable of producing stable solutions. The sensitivity of HPROM load balancing, accuracy, and computational cost savings to a variety of model parameters were rigorously investigating, resulting in the following insights:

\begin{enumerate}
    \item Randomized sampling, although simple and inexpensive to implement, generally resulted in greater memory consumption and more floating-point calculations than GNAT or eigenvector-based sampling. This is due to a wider spatial distribution of sampling points, which demands a commensurately greater number of auxiliary cells for computing fluxes, gradients, and vertex state reconstructions.
    \item On the other hand, randomized sampling produced computational meshes which require comparatively few (often zero) MPI communications. This is in contrast to GNAT sampling and eigenvector-based sampling, which often required a significant number of MPI communications. Eigenvector-based sampling generally required fewer MPI communications than GNAT sampling.
    \item Gappy POD HPROMs were able to produce robust, accurate reconstructions of the trial basis training window, although the accuracy deteriorated at extremely low sampling rates. The hyper-reduced PROMs, along with careful load balancing, were able to easily achieve over three orders of magnitude speedup in core time over the FOM in some instances.
    \item The original GNAT sampling procedure proposed tends to generate small sample meshes, but produces very inaccurate HPROMs compared to those generated by the fine-grained variant. While the original algorithm is significantly less expensive to evauate in the offline stage, the deterioration in accuracy and stability is extreme.
    \item Interestingly, random sampling tends to outperform both GNAT greedy sampling algorithms, and consumes a fraction of the offline computational cost. At very low sampling rates however, random sampling also tends to produce unstable HPROMs.
    \item On the whole, the eigenvector-based sampling strategy appeared to perform the best across most metrics of interest. Among the investigated strategies, it required the lowest memory consumption and fewest floating-point calculations, outperformed GNAT sampling in the number of required MPI communications, and generally produced the lowest average solution error for a given sampling rate across a range of time step sizes. The last point was especially true at extremely low sampling rates. Here, HPROMs using eigenvector-based sampling produced comparable error to that of the unsampled PROM, whereas HPROMs using random and GNAT sampling were highly erroneous or even unstable.
    \item The above point is not true for the nine-element combustor, for which greedy sampling algorithms (including eigenvector-based sampling) failed to generate stable HPROMs. Random sampling was capable of generating modest speedups while retaining simulation accuracy. Future work must address similar systems which are characterized by advection occurring at vastly different time scales.
    \item Increased time step sizes were able to push speedup further, though there is (unsurprisingly) a significant error increase for unsampled PROMs at much larger time steps. Interestingly, the accuracy of hyper-reduced PROMs remained fairly consistent at larger time steps.
\end{enumerate}

The work presented in this chapter represents the first successful experiments of hyper-reduced projection-based ROMs of this size and physical complexity to date, to the best of the author's knowledge. These findings establish useful heuristics for future practitioners to aid the development of accurate and scalable HPROMs for systems exhibiting similarly challenging phenomena including strong gradients, propagating waves, exceptional stiffness, and extreme scale disparity.