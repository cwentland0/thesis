\section{Conclusions}

This chapter highlights the open-source package PERFORM, developed by the author to assist the community in implementing and applying novel ROM methods for a challenging class of advection-dominated reacting flows. The flexible ROM API minimally coupled with a one-dimensional reacting compressible flow solver enables rapid prototyping and performance assessment for systems beyond the standard toy problems. 

The utility of PERFORM is demonstrated by analyzing ROM accuracy in making parametric predictions for an acoustically-forced model premixed flame, specifically investigating traditional linear subspace PROMs, deep autoencoder non-linear manifold PROMs, and non-intrusive LSTM ROMs. Both the linear and non-linear intrusive PROMs exhibit terrible performance, despite the autoencoder's excellent representation of the solution manifold at extremely low latent dimensions. On the other hand, the non-intrusive LSTM ROM displayed remarkable predictive capabilities in modeling both the average flame speed and upstream acoustic content. However, this is only true when the LSTM is trained on latent variable trajectories generated by the neural network autoencoder. Those LSTMs trained on POD modal coefficients failed to accurately represent the transient flame solution, indicative of the general inability to generate a low-dimensional and linear representation of sharp gradients and propagating waves.

Although similar results have been demonstrated for simpler canonical problems (Burgers' equation, shallow water equations), the ability to quickly implement novel ROM methods like those discussed above may be an invaluable tool for assessing their practical viability. In this sense, PERFORM stands as both a lower barrier to entry for ROM practitioners and a challenge to tackle more difficult modeling problems.