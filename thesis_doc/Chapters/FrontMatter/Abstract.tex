\chapter{Abstract}

This thesis investigates the development and application of \textit{projection-based reduced-order models} (PROMs) to mitigate the exorbitant computational cost of high-fidelity numerical simulations of complex systems. Traditionally, PROMs operate by learning a low-dimensional representation of the system state from a small amount of high-fidelity simulation data, projecting the governing equations onto a low-dimensional subspace, and evolving the resulting system on the low-dimensional manifold inexpensively. Although PROMs have been applied in industrial problems, successes are largely restricted to linear and elliptic/parabolic systems such as those describing solid mechanics, heat transfer, and diffusion-dominated flows. For advection-dominated and highly non-linear flows, classical PROMs are found to be deficient in reliably generating robust and accurate {\em predictions} of flows featuring multi-scale and multi-physics phenomena. Further, PROMs of non-linear systems require \textit{hyper-reduction} methods to achieve significant computational cost savings, and such approaches have yet to be rigorously investigated in stiff and chaotic  flow problems.

The methods developed in this thesis are motivated by and applied to complex reacting flows, with a particular emphasis on rocket combustion.
Despite decades of research in the design of rocket combustors, their development often requires long and expensive experimental test campaigns to ensure operational performance and safety. High-fidelity numerical simulation of high-pressure turbulent combustion requires enormous computational power which is not affordable in industrial design settings. 
This work advances the construction of accurate, robust, and scalable PROMs for this challenging class of problems.

This work evolves from the recent model-form preserving least-squares with variable transformation (MP-LSVT) method, which derives the ROM using a least-squares procedure, and simulates the dynamics with respect to an alternative state representation. This approach exhibits greatly improved accuracy and stability over classical PROM methods for reacting flow simulations. The method is derived here, and a number of critical nuances -- which are often not well-documented in the literature -- are detailed at length, including data preparation, extensions to non-linear manifolds, least-squares weighting, hyper-reduction regression basis and sample mesh construction, and online model adaptation. These techniques are then applied to a number of challenging multi-scale and reacting flow systems.

First, an open-source framework for implementing novel ROM approaches for 1D reacting flows, named the Prototyping Environment for Reacting Flow Order Reduction Methods (PERFORM), is outlined. Developed exclusively by the author, this package is used to conduct a critical examination of several novel neural network ROM approaches is conducted for a model premixed flame case. This approach exhibits utility in enabling accurate representations of flows characterized by sharp gradients and propagating waves. Further, non-intrusive neural network ROM approaches (i.e. those which do not require access to the numerical solver) are shown to greatly outperform comparable classical intrusive PROM methods. However, analysis of the cost of training these neural network models reveals that they are hardly a time-efficient solution compared to equivalent linear approximations.

Moving to more practical flows, scalable hyper-reduced PROMs are developed within a massively parallel compressible reacting flow solver, and demonstrated for a 2D transonic flow over an open cavity, a 3D single-element rocket combustor, and a 3D nine-element rocket combustor. The effects of the sample mesh and hyper-reduction approximation dimension on PROM performance, computational cost savings, and load balancing is probed at length. Recent algorithms for selecting sample points are shown to consistently generate accurate models, while some methods used in the classical PROM literature are shown to generate unstable solutions. Over four orders of magnitude computational costs savings, while retaining simulation accuracy, are realized for the single-element rocket combustor case. Further, the nine-element rocket combustor experiment represents the largest and most physically-complex system investigated to date, involving extreme stiffness and nearly 250 million degrees of freedom. It is shown that the MP-LSVT method is capable of accurately reconstructing the flow solution on-line. However, the ultimate goal of PROMs is truly generalizable, predictive models. To this end, analyses are conducting for a recent adaptive PROM approach, revealing that future-state and parametric predictions are achievable for very long time horizons though much work remains to be done to make such adaptive PROMs robust and efficient for such large-scale and extremely complex systems.

Finally, best practices for the development and application of PROMs are documented. Insight is provided on  centering and scaling high fidelity data snapshots. It is shown that local temperature limiters, although \textit{ad hoc}, are critical in ensuring the stability of PROMs for combusting flows. Additionally, simple least-squares-based residual weighting approaches are shown to promote long-time PROM accuracy for problems exhibiting extreme scale disparities. These guidelines will hopefully inform future PROM practitioners and help mitigate costly trial-and-error efforts. In summary, this work shows that novel projection-based reduced-order models offer an attractive means to leverage an ever-growing ecosystem of numerical and experimental data to generate accurate and low-cost solutions.