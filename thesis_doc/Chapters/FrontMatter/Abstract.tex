\chapter{Abstract}

%Despite decades of research in the design of liquid-propellant rocket combustors, their development often requires long and expensive experimental test campaigns to ensure operational performance and safety. To this day, high-frequency combustion instability in rocket engines remain an elusive and destructive phenomenon that cannot be reliably predicted by low-cost modeling techniques. High-fidelity numerical simulation of high-pressure turbulent combustion requires enormous computational power which is unavailable to industry practitioners, and thus plays a minimal role in rocket combustor design. In the era of big data, however, novel machine learning techniques propose generalizable models which might inform the engineering design process, at the cost of only a small number of high-fidelity simulations or experiments.

This thesis investigates the use of \textit{projection-based reduced-order models} (PROMs) to mitigate the exorbitant cost of high-fidelity numerical simulations. {\color{red} Explain PROMs in a sentence : dim reduction/projetion}}.  Although PROMs have been successfully applied to industrial problems, such cases are largely restricted to linear and elliptic/parabolic systems such as those describing solid mechanics, heat transfer, and diffusion dominated flows. For advection-dominated and highly non-linear flows, classical PROMs are found to to be deficient in reliably generating robust and accurate models of flows featuring multi-scale and multi-physics phenomena. Further, PROMs of non-linear systems require \textit{hyper-reduction} methods to achieve significant computational cost savings, and such approaches have yet to be rigorously investigated in stiff and chaotic  flow problems.

The methods developed in this thesis are motivated by - and applied to - complex reacting flows, with a particular emphasis on rocket combustion.
Despite decades of research in the design of liquid-propellant rocket combustors, their development often requires long and expensive experimental test campaigns to ensure operational performance and safety. %To this day, high-frequency combustion instability in rocket engines remain an elusive and destructive phenomenon that cannot be reliably predicted by low-cost modeling techniques.
High-fidelity numerical simulation of high-pressure turbulent combustion requires enormous computational power which is unavailable to industry practitioners, and thus plays a minimal role in rocket combustor design. 
This  work advances the construction of accurate, robust, and scalable PROMs for this challenging class of problems.

Our work revolves around the  model-form preserving least-squares with variable transformation (MP-LSVT) method. {\color{red} Explain MP-LSVT} in one sentence}. The method is derived here, and a number of critical nuances -which are often absent in the literature - are detailed at length, including data preparation, extensions to non-linear manifolds, least-squares weighting, hyper-reduction regression basis and sample mesh construction, and online model adaptation. These techniques are then applied to a number of challenging multi-scale and reacting flow systems.

First, a critical examination of several novel neural network ROM approaches is conducted for a model premixed flame case, showing  utility in enabling  accurate representations of flows characterized by sharp gradients and propagating waves. Further, non-intrusive neural network ROM approaches (i.e. those which do not require access to the numerical solver) are shown to greatly outperform comparable classical intrusive PROM methods. However, a detailed analysis of the cost of training these neural network models reveals that they are hardly an time-efficient solution compared to equivalent linear approximations. This work is enabled by the Prototyping Environment for Reacting Flow Order Reduction Methods (PERFORM), an open-source framework for developing novel ROM approaches for 1D reacting flows.

Moving to more practical flows, scalable hyper-reduced PROMs are developed for a 2D transonic flow over an open cavity and a 3D single-element rocket combustor. The effects of the sample mesh and hyper-reduction approximation dimension on PROM performance, computational cost savings, and load balancing is probed at length. Certain recent algorithms for selecting sample points are shown to consistently generate accurate models, while some methods used in the classical PROM literature are shown to generate unstable solutions. Over four orders of magnitude computational costs savings, while retaining simulation accuracy, are realized for the rocket combustor case.

The capabilities of PROMs are ultimately tested for arguably the largest and most physically-complex system investigated to date, a nine-element rocket combustor described by nearly 250 million degrees of freedom. It is shown that the MP-LSVT method is capable of accurately reproducing the training data. However, the ultimate goal of PROMs is generalizable, predictive models. To this end, analyses are conducting for a recent adaptive PROM approach, revealing that future-state and parametric predictions are achievable, though much work remains to be done to make such adaptive PROMs efficient for such large-scale systems.

Finally, a number of topics related to data preparation, unsteady PROM solutions, and are explored at length. Among other insights, it is shown that local temperature limiters, although \textit{ad hoc}, are critical in ensuring the stability of PROMs for combusting flows. Additionally, simple least-squares-based residual weighting approaches are shown to promote long-time PROM accuracy for problems exhibiting extreme scale disparities. These lessons in best practices will hopefully inform future PROM practitioners and help mitigate costly trial-and-error efforts. In summary, this work shows that  novel projection-based reduced-order models offer an attractive means to leverage an ever-growing ecosystem of numerical and experimental data to generate accurate and low-cost  solutions.