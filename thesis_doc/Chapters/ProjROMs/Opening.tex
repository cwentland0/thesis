We now turn to an overview of classical projection-based reduced-order model (PROM) methods, along with derivations for the recent MP-LSVT method, hyper-reduction of PROMs for non-linear systems, and ROM adaptation methods. For the duration of this chapter, we generalize discussion to ROMs of any time-dependent, non-linear, hyperbolic conservation equations, given by the semi-discrete residual
%
\begin{equation}\label{eq:FOM}
    \ode{\consVec}{\timeVar} - \rhsFunc{\consVec, \; \timeVar} = \zeroVec, \quad \consVec^0 = \consFunc{\initTime}
\end{equation}
%
Here, $\funcMap{\consVec}{\nonnegReals}{\ROne{\numDOF}}$ is the conserved state, $\timeVar \in \nonnegReals$ is the physical time, and $\funcMap{\rhsVec}{\ROne{\numDOF} \times \nonnegReals}{\ROne{\numDOF}}$ is a general function which is non-linear in the conservative state $\consVec$. In the context of the Navier--Stokes equations, this non-linear function constitutes the spatially-discretized fluxes, sources, body forces, and boundary conditions, and the number of degrees of freedom $\numDOF$ may be $\bigO{1\text{e}6-1\text{e}9}$ for simulations of practical engineering systems. We further define the fully-discrete non-linear residual $\funcMap{\resVec}{\ROne{\numDOF} \times \nonnegReals}{\ROne{\numDOF}}$ (after Eq.~\ref{eq:FOM} has been temporally discretized) as
%
\begin{equation}
    \resFunc{\consVec^{\timeIdx}, \; \timeVar^{\timeIdx}} \defEq \consVecDt^{\timeIdx} - \rhsFunc{\consVec^{\timeIdx}, \; \timeVar^{\timeIdx}} = \zeroVec,
\end{equation}
%
where $\timeIdx \in \mathbb{N}_0$ is the discrete time step index at time $\timeVar^{\timeIdx}$, and $\consVecDt^{\timeIdx}$ is the temporal discretization operator (e.g. forward Euler, BDF).

Reduced-order models have a long history of successful applications in linear and elliptic systems. We direct the reader to the review paper by Benner \textit{et al.}~\cite{Benner2015} for a discussion of PROMs for linear parametric dynamical systems, and the text by Hesthaven \textit{et al.}~\cite{certRedBasisBook} for a complete background on the reduced-basis method for parametrized differential equations. These methods are notable for their extremely well-defined ROM error measures, which enable rigorous applications in uncertainty quantification and control systems. However, such measures do not extend to general non-linear, hyperbolic systems; error bounds for such systems are usually ill-defined and impossible to compute \textit{a priori}. As such, we will not discuss ROMs for linear systems nor error bounds for ROMs of non-linear systems.