In the author's personal experience, there is often a certain opacity in the projection-based ROM literature on the subject of data preparation and robustness controls. In many cases this is simply due to the fact that many ROM studies deal with systems which are governed by a single state variable (Burgers' equation, Kuramoto--Sivashinsky equation), are governed by state variables that are naturally of similar magnitudes (shallow water equations, Euler equations for Sod shock tube), or are commonly non-dimensionalized in open-source or commercial solvers (incompressible Navier--Stokes). In such situations, the dimensionality of the state variables is either irrelevant or has little effect on the accuracy of state representations and the solution of the ROM equations. Sometimes, however, small but crucial details of data preparation or the ROM solution are left out as they are considered self-evident to the authors (as has been discovered anecdotally by this author and colleagues). To date, the work by Parish and Rizzi~\cite{Parish2022} provides the most explicit description and comprehensive study of ensuring a dimensionally-consistent POD formulation and ROM solution by careful choice of inner products. Although the focus of their work is the compressible Euler equations, they provides broadly-meaningful insights for dimensional dynamical systems.

All but one numerical experiment in this thesis deals with high-pressure, compressible reacting flows, which are characterized by both vastly disparate magnitudes in the system state and an inability to readily non-dimensionalize the governing equations. As such, careful data preparation and unsteady solution robustness controls are extremely important in enabling the stable and accurate solution of projection-based reduced-order models. Further, we wish to be as transparent about every element of the ROM construction process to help enable successful ROMs for future researchers working with similarly-complex systems. We detail some of the key elements in this process that are explored in the numerical experiments which follow.