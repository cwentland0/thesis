\section{Summary and Insights}

Chapters~\ref{chap:FlowModeling}--\ref{chap:HPROMS} provided a detailed background on modern projection-based ROMs in the context of numerical models of compressible reacting flows. Classical projection methods, namely Galerkin and LSPG projection, were contrasted against the recent MP-LSVT method, which has previously been shown (and was shown in this thesis) to be superior to classical methods in modeling reacting flows. Important nuances in the development of scalable hyper-reduced PROMs of non-linear PDEs were detailed, including their theoretical formulation, several sample mesh selection algorithms, gappy POD regressor calculation, and load balancing. Outlines of online basis and sample mesh adaptation approaches were also provided. 

Chapter~\ref{chap:TransientFlame} investigated a 1D freely-propagating, acoustically-forced model premixed flame. These calculations were enabled by PERFORM, an open-source Python framework for developing novel ROMs for reacting flows. While the 1D model flame is an extremely simple model of advection-dominated reacting flow, intrusive linear PROMs failed outright in both reconstructing training data and predicting unseen data. The ability of a linear trial space in approximating sharp gradients and traveling waves was shown to be very poor. Alternatively, modern neural network approaches were shown to be capable of approximating the solution very accurately. Deep autoencoder neural networks enabled efficient dimension reduction with extremely small latent space dimensions, inducing projection error which was orders of magnitude lower than that observed for comparable linear trial spaces. However, the integration of these autoencoders in non-linear manifold PROMs proved less encouraging, generating solution trajectories for unseen outlet pressure forcing frequencies which often failed to accurately predict the upstream acoustic content or flame speed. Further, these non-linear manifold PROMs were exorbitantly computationally-expensive, due largely to the evaluation of neural network Jacobians by automatic differentiation. To mitigate this online cost, non-intrusive ROMs were presented using recurrent neural networks, specifically LSTMs, to model the time evolution of the latent variables. This circumvented the need to repeatedly evaluate the governing equations or compute Jacobians and projections, drastically cutting down online costs. Further, the non-intrusive ROMs generated excellent online predictions for training and unseen data alike. The apparent robustness and predictive accuracy of these non-intrusive ROMs are appealing, though the computational cost of training these neural network ROMs was shown to be exceptionally high.

Chapter~\ref{chap:BestPractices} stepped back to take a big-picture examination of some of the underlying data preparation and online robustness control methods which enabled the successful results presented in previous chapters. All results presented were computed for the truncated single-element rocket combustor presented in Chapter~\ref{chap:CavityAndCVRC}. Several data centering and scaling methods were compared in generating accurate trial bases and subsequent online PROMs, showing that {\color{red}XXXXXXXXX}. Residual weighting, loosely associated with the more familiar concepts of non-dimensionalization and preconditioning, was also shown to be crucial in generating robust PROMs. In particular, {\color{red}XXXXX}. Finally, the importance of physics-informed temperature limiters was investigated, first by showing that the ringing effects near sharp gradients, a characteristic of linear representations of the system state, can sometimes lead to non-physical solutions such as negative temperatures or densities. Temperature limiters, or ``clipping'' functions, were then shown to be a simple and effective solution for preventing an unstable solution in such situations.

% Non-linear manifold ROMs are amazing, but cost is pretty atrocious
%   - non-intrusive methods remarkably more stable, accurate, generalizable than intrusive ROMs.
% Residual-minimizing ROMs are a must, already know that
% MP-LSVT confirmed to be better than LSPG for reacting flows
% Sampling algorithm is key for HPROMs:
% Adaptive ROMs enable true predictive accuracy, but marginal speedup and no memory savings
%   - Need online load balancing
% 