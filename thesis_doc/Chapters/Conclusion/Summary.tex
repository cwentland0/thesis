\section{Summary and Insights}

Chapters~\ref{chap:FlowModeling}--\ref{chap:HPROMS} provided a detailed background on modern projection-based ROMs in the context of numerical models of compressible reacting flows. Classical projection methods, namely Galerkin and LSPG projection, were contrasted against the recent MP-LSVT method, which was later shown in this thesis to be superior to classical methods in modeling reacting flows. Important nuances in the development of scalable hyper-reduced PROMs of non-linear PDEs were detailed, including their theoretical formulation, several sample mesh selection algorithms, gappy POD regressor calculation, and load balancing. Outlines of online basis and sample mesh adaptation approaches were also provided.

Chapter~\ref{chap:TransientFlame} investigated a 1D freely-propagating, acoustically-forced model premixed flame. These calculations were enabled by PERFORM, an open-source Python framework for developing novel ROMs for reacting flows. While the 1D model flame is an extremely simple model of advection-dominated reacting flow, intrusive linear PROMs failed outright in both reconstructing training data and predicting unseen data. The ability of a linear trial space in approximating sharp gradients and traveling waves was shown to be very poor. Alternatively, modern neural network approaches were shown to be capable of approximating the solution very accurately. Deep autoencoder neural networks enabled efficient dimension reduction with extremely small latent space dimensions, inducing projection error which was orders of magnitude lower than that observed for comparable linear trial spaces. However, the integration of these autoencoders in non-linear manifold PROMs proved less encouraging, generating solution trajectories for unseen outlet pressure forcing frequencies which often failed to accurately predict the upstream acoustic content or flame speed. Further, these non-linear manifold PROMs were exorbitantly computationally-expensive, due largely to the evaluation of neural network Jacobians by automatic differentiation. To mitigate this online cost, non-intrusive ROMs were presented using recurrent neural networks, specifically LSTMs, to model the time evolution of the latent variables. This circumvented the need to repeatedly evaluate the governing equations or compute Jacobians and projections, drastically cutting down online costs. Further, the non-intrusive ROMs generated excellent online predictions for training and unseen data alike. The apparent robustness and predictive accuracy of these non-intrusive ROMs are appealing, though the computational cost of training these neural network ROMs was shown to be exceptionally high.

Chapter~\ref{chap:HPROMResults} rigorously investigated the construction of accurate and scalable HPROMs for highly non-linear fluid flow systems, applying the techniques outlined in Chapter~\ref{chap:HPROMS} to a 2D transonic flow over an open cavity, a 3D single-element model rocket combustor, and a multi-element laboratory rocket combustor. To the best of the author's knowledge, this last experiment represents the largest and one of the most physically-complex PROMs in the literature to date. The effects of the sampling algorithm, sample mesh size, and gappy POD regressor dimension were examined. In particular, these results revealed that traditional sampling algorithms, such as random sampling or GNAT sampling (an extension of DEIM greedy sampling), were generally unable to generate significant computational speedups without inducing unacceptable loss of stability and accuracy. The eigenvector-based sampling proposed by Peherstorfer~\cite{Peherstorfer2020}, on the other hand, was capable to constructing a sample mesh which enabled over four orders of magnitude cost savings while retaining excellent model accuracy. However, the greedy sampling algorithms were shown to incur significant offline computational cost, albeit only for relatively large sample meshes.

Chapter~\ref{chap:AdaptiveResults} investigated limits of PROMs in making future state predictions for the two 3D rocket combustor cases. In order to construct truly generalizable models, basis and sample mesh adaptation methods were applied to these problems. Successful future-state predictions were achieved, retaining stability and accuracy for time windows two orders of magnitude  larger than the training datasets. The results presented were seen to offer an order of magnitude in cost savings, compared to many orders of magnitude savings achieved with static basis ROMs. This reveals a strong need for advanced online greedy sampling and load balancing techniques to minimize sample mesh sizes and MPI communication overhead.

Chapter~\ref{chap:BestPractices} stepped back to take a big-picture examination of some of the underlying data preparation and online robustness control methods which enabled the successful results presented in previous chapters. All results presented were computed for the truncated single-element rocket combustor presented in Chapter~\ref{chap:HPROMResults}. Several data centering and scaling methods were compared in generating accurate trial bases and subsequent online PROMs, showing that centering training datasets enhances accuracy significantly. Residual weighting, loosely associated with the more familiar concepts of non-dimensionalization and preconditioning, was also shown to be crucial in generating robust PROMs. In particular, computing residual scaling from uncentered snapshots of the conservative variables proved effective in enhancing long-term PROM accuracy. Finally, the importance of physics-informed temperature limiters, variable transformations, and efficient sample mesh computations was discussed, with recommendations for specific applications to reacting flow simulations.
% first by showing that the ringing effects near sharp gradients, a characteristic of linear representations of the system state, can sometimes lead to non-physical solutions such as negative temperatures or densities. Temperature limiters, or ``clipping'' functions, were then shown to be a simple and effective solution for preventing an unstable solution in such situations.

The sum of these results imply that projection-based reduced-order models are very nearly capable of producing efficient and generalizable data-driven models for large-scale multi-physics systems. The MP-LSVT method, improved sampling algorithms, and online adaptation approaches have been shown here to be effective solutions to many of the challenges previously faced by the PROM community. However, this work has also exposed several key issues which remain to be solved, particularly in efficient online adaptation methods. Further, novel neural network ROM approaches do not appear to be a cost-effective solution to this model generalization problem, as they are shown to be onerously expensive to train even for simple one-dimensional reacting flow problems. On the whole, however, these results are greatly encouraging and suggest many interesting paths for future research, as detailed below.