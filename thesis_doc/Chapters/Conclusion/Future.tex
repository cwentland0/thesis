\section{Future Work}

Developing data-driven models for reacting flows is a massive undertaking, and this thesis has only investigated a small portion of possible research paths. Based on my experience, I suggest a few key areas that might build off this work, restricting discussion to projection-based reduced-order models. Roughly in order of descending difficulty (in my opinion), these are:

\begin{enumerate}
    \item PROMs with advanced chemistry and multi-phase models: The chemistry models utilized in this thesis are fairly rudimentary compared to those used by contemporary researchers in turbulent combustion. Further, the assumption of gaseous propellants ignores important spray physics in liquid-propellant rocket engines. The sensitivity of PROMs in predicting complex combustion physics, such as radiation, soot generation, wall heat transfer, and spray atomization is a logical step in advancing PROMs to practical engineering systems including gas turbines, automotive engines, and jet engines.
    \item PROMs for cylindrical multi-element rocket engines: A cylindrical single-element combustor and a linear multi-element combustor were investigated in this thesis. In both, dominant system acoustics act largely in one direction. Cylindrical multi-element combustors, which are standard for industrial rocket engines, often experience complex rotational acoustic modes. The ability of PROMs to accurately model interactions between propellant injectors arranged in such configurations would help establish their utility for practical applications, rather than for the laboratory-scale experiments presented here.
    \item Online adaptive sampling algorithms: An extremely naive sampling metric for online sample mesh adaptation was used for the adaptive HPROMs presented in Chapter~\ref{chap:AdaptiveResults}, simply measuring the absolute discrepancy between the predicted FOM and PROM state at a given iteration. Although this approach was capable of making accurate future-state predictions, it required very large sample meshes to achieve a stable solution, and hence produced minimal computational cost savings. As seen is Chapter~\ref{chap:HPROMResults}, alternative greedy algorithms are capable to producing accurate reconstructions on small sample meshes, but are far too computationally-expensive to compute during online calculations. Developing efficient sampling adaptation metrics will be crucial for enabling extremely fast adaptive HPROMs.
    \item Comparison of linear adaptive PROMs and non-linear manifold ROMs: I had hoped to include linear adaptive HPROM results for the one-dimensional model premixed flame in Chapter~\ref{chap:TransientFlame}. The key difficulty in modeling that system with a linear representation was the poor approximation of sharp gradients under strong advection. Although non-linear manifold ROMs drastically improved this representation, they are extremely expensive to train. As linear adaptive PROMs incur negligible training cost, but have been shown to accurately predict advection-dominated flows~\cite{WayneIsaacTanUy2022}, a direct comparison against non-linear manifold ROMs might reveal that an adaptive linear representation is both accurate and cost-effective.
    \item Cost accounting for large-scale ML ROMs: A very superficial cost accounting for a few non-linear manifold ROMs, including offline training and online evaluation time, was presented for a simple one-dimensional model premixed flame in Chapter~\ref{chap:TransientFlame}. Even for such a low-dimensional system, the offline cost is orders of magnitude greater than that for linear PROMs. This is only emphasized for larger, more practical engineering systems, where the high training cost of deep neural networks in increased exponentially due to the curse of dimensionality. An honest evaluation of this computational burden might inform future developments in data-driven ROMs.
\end{enumerate}