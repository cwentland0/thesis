\section{Variable Transformations}

Throughout the results presented in the above chapters, it is clear that the MP-LSVT method is capable of generating stable and accurate PROMs for reacting flow systems. Although the MP-LSVT is a theoretically simple modification of the well-established LSPG projection, the effect of this change on PROM performance is profound. Although the two methods performed equivalently for the non-reacting 2D transonic cavity flow in Section~\ref{sec:cavity}, MP-LSVT was shown in Section~\ref{sec:cvrc} to preserve PROM stability for the 3D CVRC while LSPG quickly became unstable. This has similarly been observed in the original development of the MP-LSVT method by Huang \textit{et al.}~\cite{Huang2022}.

The core concept of the variable transformation has other, more subtle advantages. For one, choosing an alternative set of target variables allows a PROM practitioner to more confidently estimate \textit{a priori} the accuracy of approximating flow fields which are of more practical interest than the conservative variables. Calculations of POD bases or autoencoders can be appropriately weighted and trained to approximate certain fields with higher fidelity. For example, if the unsteady temperature field is of particular importance to an application, the training loss in approximating the temperature field can be scaled to ensure higher accuracy. This is not nearly as simple when training using the conservative variables, which may be involved in calculating various practical quantities of interest to varying degrees.   

The use of an alternate set of target variables also enables vastly simpler implementations of limiters such as those discussed in Section~\ref{sec:limiters}. When the conservative state is modeled, the field to be limited must first be calculated from the conservative variables, and then the conservative fields recalculated to reflect the change in the limited field. For example, to limit species mass fractions, the mass fractions would first be computed from the density-weighted mass fractions, clip the mass fraction fields, then recalculate the density-weighted mass fraction fields. This process incurs additional computational cost and modeling complexity which can be circumvented by directly modeling the field to be limited as part of the MP-LSVT target state.

For the above reasons, and due to its observed accuracy and robutness, the MP-LSVT method for PROMs of reacting flows can be thought of as a best practice in and of itself. Whether similar improvements can be realized for other dynamical systems remains to be seen. Further, a significant amount of work remains to determine whether there are additional target flow variable sets which result in even better PROMs than those investigated using the primitive variables.