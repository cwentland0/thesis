\section{Centering and Scaling}\label{sec:centerScale}
%
As previously defined, the general (linear or non-linear) low-dimensional state representation for the conserved and target states are defined, respectively, as
%
\begin{align}
	\consVecRom &\defEq \consVecCent + \consScale \nnLayer_{\decoderVar,\consVar}\left(\consVecCoef\right), \\
	\primVecRom &\defEq \primVecCent + \primScale \nnLayer_{\decoderVar,\primVar} \left(\primVecCoef\right),
\end{align}
%
where, again, the functions $\nnLayer_{\decoderVar,\consVar}\left(\consVecCoef\right) = \consTrial \consVecCoef$ and $\nnLayer_{\decoderVar,\primVar}\left(\primVecCoef\right) = \primTrial \primVecCoef$ for a linear trial space. This decomposition by centering ($\consVecCent$, $\primVecCent$) and the scaling ($\consScale$, $\primScale$) operation is often referred to as \textit{feature scaling} in the machine learning community, whereby the training datasets are constructed (as noted in Eqs~\ref{eq:consSnapMat} and~\ref{eq:primSnapMat}) by
%
\begin{align}
	\consDataMatUns &= \left[ \consScaleInv\left[\consFunc{\initTime} - \consVecCent\right], \; \hdots \; , \; \consScaleInv\left[\consFunc{\finalTime} - \consVecCent \right] \right], \\
	\primDataMatUns &= \left[ \primScaleInv\left[\primFunc{\initTime} - \consVecCent\right], \; \hdots \; , \; \primScaleInv\left[\primFunc{\finalTime} - \primVecCent \right] \right].
\end{align}
%
The choice of $\consVecCent$/$\consScale$ and $\primVecCent$/$\primScale$ has a measurable influence on the accuracy of the above approximations, particularly for variables of extremely disparate magnitudes. Before demonstrating this fact, several popular methods of centering and scaling are outlined and compared.

\paragraph*{Centering}\mbox{}\\
%
With the exception of centering for min-max scaling, all centering methods described here are considered to be spatially-variant, i.e. $\consVecCent \defEq \consVecCent(\spatialVec)$, $\primVecCent \defEq \primVecCent(\spatialVec)$. Each is described in turn.

\begin{enumerate}
	\item \textit{Initial condition}: Centering about the initial condition, i.e. $\consVecCent = \consVec(\initTime)$ or $\primVecCent = \primVec(\initTime)$, results in approximation of the unsteady state as perturbations about the initial condition. In the case of a linear trial space, this guarantees exact satisfaction of the initial conditions, as the projection of the zero vector (the initial condition subtracted by itself) is identically zero. In the case of autoencoder non-linear manifold methods, however, this is merely satisfied approximately, and near-satisfaction is encouraged by including the zero vector in the training set and initializing the non-linear manifold PROM from the encoding of the zero vector, as in~\cite{Lee2020}.

	\item \textit{Mean}: The mean field centering computes the centering vector as the arithmetic mean of the data snapshots,
	%
	\begin{align}
		\consVecCent &= \frac{1}{\numSnaps} \sum_{\timeIdx=1}^{\numSnaps} \consVec^{\timeIdx} \label{eq:meanCentCons}\\
		\primVecCent &= \frac{1}{\numSnaps} \sum_{\timeIdx=1}^{\numSnaps} \primVec^{\timeIdx} \label{eq:meanCentPrim}
	\end{align}
	%
	This concept is fairly common in the turbulence modeling community, which often seeks to accurately describe unsteady perturbations about the time-averaged field for statistically-stationary flows. In the same sense, centering data snapshots about the mean field ensures that the low-dimensional representation accurately captures these small-scale fluctuations which would otherwise be dwarfed by the mean field.

	\item \textit{Min-max}: Min-max feature scaling, composed of associated centering and scaling operations (see below for the scaling operation), has the effect of ensuring that all values in the modified dataset fall in the range $[0.0, \; 1.0]$. For the $\varIdx$th state variable (e.g. density, velocity), the centering vector is computed as
	%
	\begin{align}
		\consValCentVar &= min\left(\consDataMatVar\right), \quad \consVecCentVar\left(\spatialVec_{\dummyIdx}\right) = \consValCentVar \; \forall \; \dummyIdx \in \{1, \; \hdots, \; \numCells\} \\
		\primValCentVar &= min\left(\primDataMatVar\right), \quad \primVecCentVar\left(\spatialVec_{\dummyIdx}\right) = \primValCentVar \; \forall \; \dummyIdx \in \{1, \; \hdots, \; \numCells\}
	\end{align}
	%
	where the data snapshot matrix for the $\varIdx$th state variable is given by
	%
	\begin{align}
		\consDataMatVar &\defEq \left[ \consVecVar\left(\initTime\right), \; \hdots, \; \consVecVar\left(\finalTime\right) \right] \label{eq:consDataMat}\\
		\primDataMatVar &\defEq \left[ \primVecVar\left(\initTime\right), \; \hdots, \; \primVecVar\left(\finalTime\right) \right] \label{eq:primDataMat}
	\end{align}
	%
	The reader is directed to the min-max scaling description below for further details on this approach.
\end{enumerate}

\paragraph*{Scaling}\mbox{}\\
%
For all methods described below, the scaling matrices $\consScale$, $\primScale$ are composed of constant scalars which are specific to a given state variable (e.g., density, velocity) but are not specific to spatial location. This can be written as as
%
\begin{align}
	\consScale &\defEq diag\left(\consScaleVecVar{1}^\top, \; \hdots, \; \consScaleVecVar{\numVars}^\top \right), \quad \consScaleVecVar{\varIdx}\left(\spatialVec_{\dummyIdx}\right) = \consScaleVar{\varIdx} \; \forall \; \dummyIdx \in \{1, \; \hdots, \; \numCells\} \\
	\primScale &\defEq diag\left(\primScaleVecVar{1}^\top, \; \hdots, \; \primScaleVecVar{\numVars}^\top \right), \quad \primScaleVecVar{\varIdx}\left(\spatialVec_{\dummyIdx}\right) = \primScaleVar{\varIdx} \; \forall \; \dummyIdx \in \{1, \; \hdots, \; \numCells\}
\end{align}
%
where $\consScaleVecVar{\varIdx}$, $\primScaleVecVar{\varIdx} \in \mathbb{R}$ is the constant conservative/target scaling value for the $\varIdx$ state variable.

\begin{enumerate}
	\item $\ell^2$-norm: Scaling by the $\ell^2$-norm method is motivated by that proposed by Lumley and Poje~\cite{Lumley1997}, and is computed as
	%
	\begin{align}
		\consScaleVar{\varIdx} &= \frac{1}{\numSnaps} \sum_{\timeIdx=1}^{\numSnaps} \left\Vert \consVecVar^{\timeIdx} - \consVecCentVar \right\Vert^2 \\
		\primScaleVar{\varIdx} &= \frac{1}{\numSnaps} \sum_{\timeIdx=1}^{\numSnaps} \left\Vert \primVecVar^{\timeIdx} - \primVecCentVar \right\Vert^2
	\end{align}
	This has the effect of ensuring that all state variable vectors have an length (in the Euclidean norm) close to unity. This is closely related to the POD, which computes the distance between the data and its projection in the $\ell^2$ norm.

	\item Standardization: Data standardization is coupled with the calculation of the mean field, as described above. The scaling constants are calculated as
	%
	\begin{align}
		\consScaleVar{\varIdx} &= \sqrt{\frac{1}{\numSnaps \numCells} \sum_{\timeIdx=1}^{\numSnaps} \sum_{\dummyIdx = 1}^{\numCells} \left[\consVecVar^{\timeIdx}\left(\spatialVec_{\dummyIdx}\right) - \consVecCentVar\left(\spatialVec_{\dummyIdx}\right) \right]^2} \\
		\primScaleVar{\varIdx} &= \sqrt{\frac{1}{\numSnaps \numCells} \sum_{\timeIdx=1}^{\numSnaps} \sum_{\dummyIdx = 1}^{\numCells} \left[\primVecVar^{\timeIdx}\left(\spatialVec_{\dummyIdx}\right) - \primVecCentVar\left(\spatialVec_{\dummyIdx}\right) \right]^2}
	\end{align}
	%
	which is the standard deviation for the $\varIdx$th state variable of the dataset and $\consVecCentVar$, $\primVecCentVar$ are the mean field for the $\varIdx$th state variable as calculated in Eq.~\ref{eq:meanCentCons} and~\ref{eq:meanCentPrim}. This operation has the effect of ensuring that the centered and scaled dataset for each state variable has a mean of zero and standard deviation of one.

	\item Min-max: As stated for the min-max centering above, the min-max scaling and centering are strictly coupled, and the implementation of both results in scaling all variables such that they fall within the range $[0.0, \; 1.0]$. The scaling values are computed as
	%
	\begin{align}
		\consScaleVar{\varIdx} &= max\left(\consDataMatVar\right) - min\left(\consDataMatVar\right) \\
		\primScaleVar{\varIdx} &= max\left(\primDataMatVar\right) - min\left(\primDataMatVar\right)
	\end{align}
	%
	where the data matrices are given as in Eqs.~\ref{eq:consDataMat} and~\ref{eq:primDataMat}.

	Although min-max scaling is fairly common within the machine learning community, and is a simple method of ensuring that the data are roughly the same order of magnitude, this scaling has the effect of emphasizing outliers, which thus define the maximum and minimum bounds of the state variables.
\end{enumerate}
