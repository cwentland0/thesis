\subsection{Simulation of Reacting Fluid Flows}

Computer simulations of physical processes have existed since the World War II era, in which the development of early supercomputers and researchers' access to them expanded greatly, all in pursuit of the atomic bomb. Broadly, the purpose of numerical simulations is to approximately reconstruct a physical process which has no tractable analytical solution and is either impossible or cost-prohibitive to experimentally measure. Naturally, rocket engines seem an ideal candidate for simulation due to their physical complexity and high cost of production. We break this process into its two major components: the modeling of fluid flows and the modeling of chemical reactions.

Numerical fluid modeling, or computational fluid dynamics (CFD), is a fairly mature field based in the modeling of the Navier--Stokes equations (detailed in Sec.~\ref{sec:govEqs}), a set of partial differential equations which dictate the conservation of mass, momentum, and energy of a fluid flow. Except in special circumstances where a fluid cannot be treated as a continuum (such as nanofluidics and rarefied flows), the Navier--Stokes equations are a reliable description of fluid flows. However, analytical solutions for these equations are limited to extremely simple scenarios, such as Poiselle pipe flow or the Taylor--Greene vortex, where elements of the Navier--Stokes equations are neglected or the system exhibits some symmetry. In all other scenarios, certainly in any application of practical significance, analytical solutions are impossible and numerical methods must be used to obtain approximate solutions. In general, these methods amount to discretizing the spatial domain via methods such as finite-difference, finite-volume, finite-element, or discontinuous Galerkin methods, and marching the system forward in time with temporal discretization methods. These simulations are regularly employed in the design of aircraft, automobiles, watercraft, and munitions. However, these models generally grow rapidly in cost with their spatial and temporal resolution. Many very important physical phenomena, such as turbulence, require extremely small spatio-temporal resolutions to capture accurately. This necessitates the use of powerful supercomputers, which are inaccessible to all but select government and academic researchers, to compute direct numerical simulations (DNS), models which resolve all characteristic spatio-temporal scales. Most CFD practitioners must instead rely on low-fidelity approximations (such as Reynolds-averaged Navier--Stokes models or large eddy simulations) to describe the effect of unresolved physics on the resolved system in order to obtain a computationally-tractable solution. While comparably inexpensive, these models often fail to predict important flow features such as the transition of laminar flow to turbulent flow, or the onset of flow separate. While such low-cost models are useful as a first-order approximation, but cannot match well-resolved simulations in their accuracy and generalizability. 

Modeling chemical reactions is also a well-studied field. Combustion, or any exothermic reaction of a fuel (often a hydrocarbon) with an oxidizer (often oxygen), is of particular interest due to its importance in energy generation, both for electricity in power plants and for thrust in jet and rocket engines. Chemical reactions are, at their core, collisions between molecules which may or may not react depending on their individual velocities, collision angle. There is an enormous number of molecules in relevant chemical reaction systems (there are 1.882$\times 10^{22}$ molecules in just one gram of oxygen), and it is impossible to model their individual behaviors. As such, we rely on statistical or empirical models to describe the bulk behavior of chemicals in a mixture. Many detailed chemical reaction mechanisms have been developed which include dozens of molecules and hundreds of reactions. These are, in general, not computationally tractable for simulations of large, complex reacting systems. Reduced mechanisms seek to condense these complex processes with fewer chemical species and reactions, but sometimes fail to match experiments under certain conditions (e.g., low-pressure, high equivalence ratio).

The combination of these two fields, reacting fluids flows, is much less understood than its component fields alone. The characteristic spatio-temporal scales of chemical reactions are smaller even than those of turbulence, placing well-resolved simulations for all but fairly simple canonical problems well out of reach of modern supercomputers. Further, the highly non-linear reaction source terms produce extremely stiff systems, making robust numerical solutions challenging. Accurately modeling combustion in the presence of turbulence has been a particular difficulty for decades, but is hotly researched thanks to the importance of turbulence in efficient, reliable, and low-emission combustion. A bevy of approximate models, including flamelet/progress variable, thickened flame, and transported PDF models, have been successfully applied to many practical combusting flow systems. However, many of these models struggle to accurately predict important features (such as ignition and extinction) in all flow configurations (such as partially-premixed flames). 

The above paragraphs illustrate two important points. First, the well-resolved simulations of combusting flows are extremely computationally expensive, largely driven by disparate and coupled spatio-temporal scales of turbulence and reactions, as well as by complex and stiff chemical kinetics. Second, although low-cost models such as RANS/LES, reduced chemical mechanisms, or flamelet models are developing and improving rapidly, they are often not generalizable to all circumstances and may not achieve significant cost reduction to be industrially-viable. This stimulates inquiry into a third path, which seeks to learn learn low-cost surrogate models of complex systems from a small number of experiments or high-fidelity simulations. Such approaches can be broadly categorized as ``data-driven  modeling.'' 