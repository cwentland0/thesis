\section{Objectives and Contributions}

As discussed above, there is a relative paucity of investigations of PROMs for extremely large-scale, highly non-linear, multi-scale and multi-physics flow problems of some engineering significance. This thesis helps to fill this gap, detailing many of the unique challenges of PROMS for reacting flow problems, exposing the limitations of the state-of-the-art projection-based ROMs, and proposing some solutions for these problems and the methods by which they can be achieved in a compute- and memory-scalable fashion. In summary, the contributions by the author detailed in this thesis are as follows:

\begin{enumerate}
    \item Sole development of PERFORM, an open-source project with which the reduced-order modeling community can prototype and test new methods for a series of benchmark problems which are significantly more challenging and informative than standard ROM toy problems. Its use in pushing the capabilities of modern ROM methods is evaluated for an acoustically-forced 1D premixed flame, investigating novel deep autoencoder and recurrent neural network ROM methods. These results, enabled by PERFORM, give insight into both the exceptional ability of neural networks in representing advection-dominated reacting flows and the excessive cost of training the same neural networks.
    \item A baseline assessment of state-of-the-art linear projection-based models for several complex multi-scale systems, showing that the prevailing state-of-the-art is ineffective. Detailed analyses of the MP-LSVT method demonstrate its superior performance for problems of greater complexity than any previously tested with the same approach.
    \item Application of projection-based reduced-order modeling for a 3D multi-injector rocket combustor with nearly 250 million degrees of freedom, wherein combustion is modeled via a 12-species and 38-reaction. This represents the first study (to the author's knowledge) of a problem of this size and physical complexity, involving a self-excited transverse combustion instability, coupled injector dynamics, and exceedingly stiff reaction kinetics. Although this system is still a vast simplification of real combustor physics, this effort is a major step forward towards demonstrating the viability of PROM methods for industrial-scale systems.
    \item Implementation of a pre- and post-processing toolchain for hyper-reduced PROMs in the open-source linear algebra toolkit PLATFORM~\cite{PLATFORM}. This enables distributed solutions for costly greedy algorithms which are required to generate stable and accurate hyper-reduced PROMs, and is designed for easy expansion with novel sampling methods. Further, these additions allow for rapid generation of full-dimensional field data from hyper-reduced PROM outputs for error measurements and visualization. This end-to-end process assists in applying hyper-reduced PROMs to extremely high-dimensional systems.
    \item Overhaul of non-scalable PROM hyper-reduction implementation in a massively-parallel combustion CFD laboratory code, enabling truly memory- and compute-scalable PROMs which do not scale with the original high-fidelity model dimension. With this redesign, true computational cost savings can be realized, and low-cost, low-memory simulations of extremely large combusting flow systems may be computed on laptops or personal workstations.
    \item Demonstrated scalability of hyper-reduction for PROMs of practical engineering systems, including 2D transonic cavity flow and a 3D rocket combustor. Over three orders of magnitude computational savings are realized, decreasing the cost from $\MC{10,000}$ CPU-hours to $\MC{10}$ CPU-hours. This appears to be the first rigorous comparison of classical sparse sampling algorithms and more recent algorithms for systems of this size and complexity, revealing stark differences in load-balancing, accuracy, and computational cost savings. Insights into other critical parameters which define the hyper-reduced PROM problem are also made, aiding future investigations into PROMs of multi-scale, multi-physics systems.
    \item Overhaul of inefficient basis and hyper-reduction sample mesh adaptation algorithms in the aforementioned massively-parallel combustion CFD code. This enables the execution of truly predictive PROMs for large-scale systems which were previously unattainable due to excessive memory consumption and inter-process communication overhead. This is demonstrated for a 3D single-element rocket combustor, for which accurate models of unseen physics beyond the training dataset are computed with dynamic trial spaces and sample meshes.
    \item A collected discussion and analysis of several important steps in developing effective PROMs for large-scale complex problems, including data preparation (centering and normalization), residual weighting for residual-minimization PROMs, limiters/clipping functions for preventing non-physical solutions, and variable transformations for PROMs of extremely stiff systems. These topics are often ignored, neglected, or glossed over in the literature, leaving future researchers to waste time learning these simple tricks of the trade. Especially for large-scale multi-scale systems characterized by propagating waves, sharp gradients, and disparate dimensional scales, these methods are shown to be crucial elements of the PROM toolchain, and findings here provide best practices for practicioners. 
\end{enumerate}

Take care to note, however, that this thesis does \textit{not} incorporate many of the complex physical phenomena present in realistic liquid-propellant rocket combustion. Due to computational constraints, none of the ``high-fidelity'' models can be considered well-resolved in space or time, they do not incorporate liquid-phase transport or reactions, and all reaction mechanisms are greatly reduced. These results are not presented as accurate reflections of rocket combustion, and highlight a common criticism of data-driven methods for complex systems: if the underlying numerical model does not reflect reality, what is the use of data-driven approaches which generate unrealistic solutions? This is a somewhat shortsighted view, ignoring the progressive nature of scientific research. Data-driven models cannot instantaneously advance from toy problems to industrial-scale systems, but rather require incremental application more difficult problems and adjustments as obstacles appear. 

It is in this frame of mind that this thesis should be viewed instead as one of many steps towards the application of PROMs to realistic rocket combustion simulations. This body of work thus demonstrates that projection-based ROMs of non-linear systems can be a practical and effective means of generating low-cost solutions to extremely challenging problems in engineering. While there remains much work to be done to make these models truly generalizable and viable for industrial applications, they pose an encouraging route of investigation for alleviating the burdensome cost of extensive experimental campaigns and massive high-fidelity simulations in engineering design and analysis.