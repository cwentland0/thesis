\section{Objectives and Contributions}

As discussed above, there is a relative paucity of investigations of PROMs for extremely large-scale, highly non-linear, multi-scale and multi-physics flow problems of some engineering significance. This thesis aims to fill some of this gap, detailing the unique challenges of PROMS for reacting flow problems, exposing the limitations of the state-of-the-art projection-based ROMs, and proposing some solutions for these problems and the methods by which they can be achieved in a compute- and memory-scalable fashion. In summary, the results presented here will attempt to provide the following information:

\begin{enumerate}
    \item A baseline assessment of state-of-the-art linear projection-based models for several complex multi-scale systems, showing that the prevailing SOA is ineffective. Detailed analyses of the  MP-LSVT method demonstrate its superior performance for problems of greater complexity than any previously tested with the same approach.
    \item Application  of projection-based reduced-order modeling for a 3D multi-injector rocket combustor with nearly 250 million degrees of freedom, the first study (to the author's knowledge) of a problem of this size and physical complexity. This represents a major step forward towards demonstrating the viability of PROM methods for industrial-scale systems.
    \item Evidence of scalability of hyper-reduction for PROMs of practical engineering systems, including two-dimensional transonic cavity flow and a three-dimensional rocket combustor. Three to four orders of magnitude computational cost savings are realized, and insights are provided into efficient construction of sample meshes which improve the stability and accuracy of the hyper-reduced PROMs.
    \item Truly predictive PROMs for a 3D single-element rocket combustor which are capable of accurately modeling unseen physics beyond the training dataset via trial basis and sample mesh adaptation. Several critical issues for the efficient implementation of adaptive PROMs are discussed. 
    \item A collected analysis of several important steps in developing effective PROMs for large-scale complex problems (e.g. residual weighting, limiters, centering/scaling, etc.). These oft-neglected topics are shown to be crucial elements of the PROM toolchain, and findings here provide best practices for practicioners. 
    \item PERFORM: an open-source project with which the reduced-order modeling community can rapidly prototype and test new methods for a series of benchmark problems which are significantly more challenging and informative than standard ROM toy problems. Its use in pushing the capabilities of modern ROM methods is evaluated for an acoustically-forced 1D premixed flame, revealing key drawbacks in some popular neural network-based approaches, and interesting opportunities in others.
\end{enumerate}

This body of work demonstrates that projection-based ROMs of non-linear systems can be a practical and effective means of generating low-cost solutions to extremely challenging problems in engineering. While there remains much work to be done to make these models truly generalizable and viable for industrial applications, they pose an encouraging route of investigation for alleviating the burdensome cost of extensive experimental campaigns and massive high-fidelity simulations in engineering design and analysis.