\section{Combustion Instabilities in Liquid-propellant Rocket Engines}

In the 21st century, the ability to launch objects into space is critical to the function of national governments, corporations, and scientific researcher organizations alike. Whereas space was previously a playground for the United States and the USSR to jostle for socio-political supremacy, it is now inseparable from the everyday lives of almost every human being. Although the average human will never personally launch into space, the impact of space vehicles, probes, and satellites is experienced in the most mundane tasks and life-critical operations. Communication satellites provide data transmission for cell phones, and GPS satellites provide real-time location and navigation services. Cameras pointed at the Earth from satellites help track storms and wildfires, and observe dangers to the environment as glaciers recede and rainforests are clear cut. Space telescopes peer into the depths of the universe to better understand our place in the cosmic soup of galaxies, suns, and planets. Space stations provide a testing ground for life to thrive beyond Earth, and enable new discoveries in medicine, agriculture, and engineering to improve life back on Earth. These extra-planetary objects accomplish incredible things, and getting them into space requires similarly incredible feats of engineering.

Since the dawn of spaceflight, launching these massive objects into space has been accomplished exclusively by chemical rocket propulsion. No other practical means of propulsion is able to generate the force required to lift payloads hundreds of miles and achieve velocities greater than 15,000 miles per hour. Greatly oversimplifying, rocket engines accomplish this by reacting chemicals which generate high-temperature and high-pressure gases, forcing these gases through a nozzle which accelerates them to supersonic speeds, and then ejecting the gases behind the engine. This ejection of high-speed gas imparts enormous force, or ``thrust,'' on the rocket: each of the five Rocketdyne F-1 engines of the Apollo missions' Saturn V rocket provided roughly 6.77 million newtons of thrust, or in President John F. Kennedy's words, ``...power equivalent to 10,000 automobiles with their accelerators on the floor.'' While air-breathing jet engines, like those which power commercial aircraft or military fighter planes, are capable of generating large thrust levels (the General Eletric GEnx can generate over 0.3 million newtons), they suffer from an unfortunate lack of air in space.

Chemical rocket propulsion comes in two dominant flavors: solid and liquid propellants. Solid propellants are fairly self-explanatory, composed of combustible reactants cured into a solid fuel ``grain,'' which sustains a highly-energetic reaction until the entire grain is consumed. Solid rocket motors can be stored at room temperature for long periods of time, and can be ignited at a moment's notice. As such, they play a dominant role in rocket-propelled munitions such as rocket artillery and missiles, like the American LGM-30 Minuteman intercontinental ballistic missile. However, solid propellant reactions are extremely difficult to control, as combustion cannot be stopped once it begins. Further, the reaction of solid propellants tends to be far less efficient than reactions between many liquid or gaseous propellants, achieving lower thrust-to-weight ratios \textcolor{red}{CONFIRM THIS}.

Due to solid propellants' aforementioned drawbacks, liquid propellants are, by far, the preferred method to power space launch vehicles. They enable much more fine-tuned thrust control: liquid engines have been designed to shut down and restart multiple times (\textcolor{red}{EXAMPLE}), throttle thrust over a variety of operating condititions (\textcolor{red}{EXAMPLE}), and actively alter the thrust vector to modify a vehicle's trajectory and orientation (\textcolor{red}{EXAMPLE}). Liquid propellants come in to additional variants: bipropellants and monopropellants. Bipropellants involve a liquid oxidizer and liquid fuel, which are mixed and atomized into a gas before combustion. Common oxidizers include liquid oxygen or dinitrogen tetroxide, and common fuels include liquid hydrogen, kerosene (RP-1), unsymmetric dimethylhydrazine, and monomethylhydrazine. Monopropellants involve a single propellant which decomposes in a highly exothermic reaction when it comes in contact with a catalyst. The most common monopropellant is hydrazine.    

\begin{itemize}
    \item Importance of launch vehicles to modern life
    \begin{itemize}
        \item Communication and GPS satellites
        \item Earth observation (meteorology, fire monitoring, mapping)
        \item Space telescopes
    \end{itemize}
    \item History of liquid rocket combustion for defense, government, and commercial
    \begin{itemize}
        \item Meaning of liquid propellant, comparison in use to solid, hybrid, electric
        \item Have to use chemical propulsion for launch, highest specific impulse
        \item Large first-stage boosters, smaller second-stage (bipropellants, LOX and LH2 or kerosene/RP-1, dinitrogen tetroxide and monomethylhydrazine or unsymmetrical dimethylhydrazine) (US F1 - Saturn, US RS-25 - Shuttle/SLS, SpaceX Merlin/Raptor - Falcon/Starship, Russian RD-107 - Soyuz, Chinese CZ-5-300 - Long March 5)
        \item Unmanned spacecraft, satellite thrusters, maneuvering thrusters (most use hypergolic or monopropellant, hydrazine)
        \item Missiles (German V-2, Chinese DF-5, Russian RSM-56 Bulava final stage, less common since liquid much harder to store)
        \item History of liquid propellant rocket engines in the United States~\cite{Sutton2003}
    \end{itemize}
    \item History of combustion instabilities in rocket engines
    \begin{itemize}
        \item ``Chugging'' in feed lines and ``buzzing'' in structural components can be fairly easily fixed
        \item High frequency, high amplitude combustion instability much harder to diagnose, might only happen in one out of a hundred tests, just had to do a ton of tests
        \item Bomb testing
        \item Problems: weld cracks, erosion/burn-through of chamber walls and injector faces, decreased thrust/impuls
        \item Mention disruption of cooling mechanisms, gas temps are twice that of melting point of steel (Sutton)
    \end{itemize}
    \item Brief overview of combustion and acoustic processes
    \begin{itemize}
        \item NOTE: Limit this to non-premixed flames. Can mention premixed combustion, like RDE, but out of scope. 
    \end{itemize}
    \item Combustion instability mechanism
    \begin{itemize}
        \item Pulsing combustion of accumulated propellants
        \item Bubbles in propellant from pressure-fed systems
        \item Resurging from film cooling (not used much any more)
        \item Explosion of droplets at supercritical pressures
    \end{itemize}
    \item Mention differences with gas turbine instabilities, unique challenges
    \begin{itemize}
        \item Combustion instabilities in gas turbine engines: operational experience, fundamental mechanisms and modeling~\cite{Lieuwen2005}
    \end{itemize}
    \item Methods of preventing/damping instabilities in rocket combustors (injector arrangement, baffles, cavities, etc.)
    \begin{itemize}
        \item A review of acoustic dampers applied to combustion chambers in aerospace industry~\cite{Zhao2015}
        \item A review of active control approaches in stabilizing combustion systems in aerospace industry~\cite{Zhao2018}
    \end{itemize}
    \item Analytical/empirical methods of predicting combustion instabilities, drawbacks
    \begin{itemize}
        \item 
    \end{itemize}
    \item Still a problem, though less so. 
    \begin{itemize}
        \item Throttled engines through disparate operating conditions
        \item Multiple restarts make problems during transients more common
    \end{itemize}
\end{itemize}