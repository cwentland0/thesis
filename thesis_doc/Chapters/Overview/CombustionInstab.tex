\section{Combustion Instabilities in Liquid-propellant Rocket Engines}

In the 21st century, the ability to launch objects into space is critical to the function of national governments, corporations, and scientific research organizations alike. Whereas space was previously a playground for the United States and the USSR to jostle for socio-political supremacy, it is now increasingly accessible to developing countries and, more broadly, inseparable from the everyday lives of almost every human being. Although the average human may never personally launch into space, the impact of space vehicles, probes, and satellites is experienced in the both mundane tasks and life-critical operations. Communication satellites provide data transmission for cell phones, and GPS satellites provide real-time location and navigation services. Cameras pointed at the Earth from satellites help track storms and wildfires, and observe dangers to the environment as glaciers recede and rainforests are cut. Space stations provide a testing ground for life to thrive beyond Earth, and enable new discoveries in medicine, agriculture, and engineering to improve life back on Earth. Space telescopes peer into the depths of the universe to better understand our place in the cosmic soup of galaxies, suns, and planets. These extra-planetary objects accomplish incredible things, and getting them into space requires similarly incredible feats of engineering.

Since the dawn of spaceflight, launching these massive objects (many satellites weight over one ton) into space has been accomplished exclusively by chemical rocket propulsion. No other practical means of propulsion is able to generate the force required to lift payloads hundreds of miles and reach velocities greater than 15,000 miles per hour. Greatly oversimplifying, rocket engines accomplish this by reacting chemicals which generate high-temperature and high-pressure gases in a combustion chamber, forcing these gases through a nozzle which accelerates them to supersonic speeds, and then ejecting the gases behind the engine (\textcolor{red}{cross section of rocket engine?}). This ejection of high-speed gas imparts enormous force, or ``thrust,'' on the rocket: each of the five Rocketdyne F-1 engines of the Apollo missions' Saturn V rocket provided roughly 6.77 million newtons of thrust. While air-breathing jet engines, like those which power commercial aircraft or military fighter planes, are capable of generating large thrust levels (the General Eletric GEnx can generate over 0.3 million newtons), they suffer from an unfortunate lack of air in space.

Chemical rocket propulsion comes in two dominant flavors: solid and liquid propellants. Solid propellants are fairly self-explanatory, composed of combustible reactants cured into a solid fuel ``grain.'' When fired, it sustains a highly-energetic reaction converting solid fuel into hot gases until the entire grain is consumed. Solid rocket motors can be stored at room temperature for long periods of time, and can be ignited at a moment's notice. As such, they play a dominant role in rocket-propelled munitions such as rocket artillery and missiles, like the American LGM-30 Minuteman intercontinental ballistic missile. However, solid propellant reactions are extremely difficult to control, as combustion cannot be stopped once it begins. Further, the reaction of solid propellants tends to be far less efficient than reactions between many liquid or gaseous propellants, achieving lower thrust-to-weight ratios.

Due to solid propellants' aforementioned drawbacks, liquid propellants are, by far, the preferred method to power space launch vehicles. They enable much more fine-tuned thrust control: liquid engines have been designed to shut down and restart multiple times (e.g., the Apollo Lunar Module reaction control system), and the thrust can be actively throttled to adjust the rocket's trajectory (e.g., SpaceX Falcon purported to throttle down to $\sim$25\% thrust). Liquid propellants come in to additional variants: bipropellants and monopropellants. Bipropellants involve a liquid oxidizer and liquid fuel, which are mixed and atomized into a gas before combustion. Common oxidizers include liquid oxygen or dinitrogen tetroxide, and common fuels include liquid hydrogen, kerosene (RP-1), unsymmetric dimethylhydrazine, and monomethylhydrazine. Monopropellants involve a single propellant which decomposes in a highly exothermic reaction when it comes in contact with a catalyst. The most common monopropellant is hydrazine. The selection of liquid propellants largely depends on the engine's mission (e.g., launch, station-keeping, orientation control) and a balance of combustion efficiency (liquid hydrogen and oxygen is the most efficient combination), cost (kerosene is cheaper than liquid hydrogen to manufacture and store), and risk to humans and the environment(monomethylhydrazine is highly toxic).

Despite many benefits, the design, manufacturing, and operation of liquid propellant rocket engines are not without major complications. They require complex pump systems to pressurize the propellants, some propellants must be stored at temperatures below \mbox{-300$^{\circ}$} F, and require intricate injection systems to mix and atomize the liquid to promote efficient combustion. A major design flaw which has hounded engine designers for decades is the possibility of \textit{combustion instabilities}.

This phrase broadly refers to any vibrations in the engine propellants and structure which can result in degraded performance, and potentially damage to or complete destruction of the engine. There are three broad categories of such instabilities: propellant feed instabilities (``chugging''), structural instabilities (``buzzing''), and resonant combustion (``screaming''). As the colloquial names may imply, these are listed in order of increasing frequency of the vibration and relative danger to the integrity of the engine. Chugging instabilities are low-frequency, low-amplitude oscillations generated by irregular injection of propellants, leading to the accumulation of unburnt reactants and subsequent explosive reaction. Buzzing instabilities result from the intermediate-frequency coupling of acoustic waves in the reacting gases and the structural components of the engine assembly. Chugging and buzzing are not always a threat to the engine's integrity, but can lead to lower, irregular performance.

The final instability, screaming, refers to a highly destructive feedback mechanism between acoustic waves in the combustion chamber and the unsteady heat release caused by the reaction of the propellants. The reaction generates heat, which raises the pressure of the reacting gases, which in turn encourages a more vigorous reaction, which generates more heat, and so on. Depending on the geometry of the combustion chamber, the composition, temperature, and feed rate of the propellants, and the ability of the engine's structure to dissipate acoustic energy, this feedback loop may amplify the acoustic waves to enormous amplitudes. These large pressure waves and the accompanying high heat-release rates may shake engine apart, or burn through the combustion chamber walls or propellant injection mechanisms. Unlike chugging and buzzing, which can be easily fixed, screaming is nearly impossible to predict and there exist no surefire methods of preventing it.

Screaming has been catalogued in engine development dating back to the 1950's in the American Thor and Atlas missile programs, and continued to plague development programs throughout the prolific USA-USSR Space Race era and beyond. \textcolor{red}{EXPAND}

Due to the extreme complexity of coupled combustion and fluid flow dynamics, paired with the complex geometry and propellant injection configurations in rocket engines, analytical methods for understanding combustion instabilities often rely on gross oversimplifications of the rocket engine physics. \textcolor{red}{EXPAND}

There are generally four methods by which high-frequency combustion instabilities may be mitigated, none of which are guaranteed to eliminate them. The first is simply modifying the operating conditions inside the combustion chamber until the problem disappears. This may include changing the chamber pressure, the propellant mixture composition, the propellant flow rates, or the propellant temperature. The second is modifying the propellant injection configuration, altering the location and angle of injection of the liquid fuel and oxidizer, until the instability is eliminated. The third is the construction of short ``baffles,'' or metal barriers which divide the propellant injection surface into smaller subsections in the hopes of increasing the resonant frequency within these cavities above a point where the acoustics might couple with the unsteady heat release. The final method is the construction of many small chambers in the combustion chamber walls which act as acoustic dampers. These help absorb acoustic energy and theoretically eliminate the growth of unstable acoustic modes. Although the latter two methods are certainly more principled than the former, all methods require a trial-and-error approach to engine design. If the combustion instability is not eliminated quickly, this process may easily lead to program cost overruns and failure to deliver the engine on schedule. \textcolor{red}{talk about F-1}

In the modern era, in which a bevy of private companies such as Blue Origin, SpaceX, Boeing, and Lockheed Martin jockey for a lucrative heavy-launch market and NASA takes a relative back seat, publicly-available data on engine development programs is scarce. However, there have been occasional hints that combustion instabilities continue to haunt their engine design programs well into the 2020's \textcolor{red}{(Tory Bruno tweet?)}. With the understanding that analytical methods of predicting instability fall short in their generalizability, and rigorous test campaigns may incur excessive costs, we now turn to the use of numerical simulations in the modeling of reacting fluid flows, with an emphasis on applications to rocket combustors.

\begin{itemize}
    \item Brief overview of combustion and acoustic processes
    \begin{itemize}
        \item NOTE: Limit this to non-premixed flames. Can mention premixed combustion, like RDE, but out of scope.
    \end{itemize}
    \item Combustion instability mechanism
    \begin{itemize}
        \item Pulsing combustion of accumulated propellants
        \item Bubbles in propellant from pressure-fed systems
        \item Resurging from film cooling (not used much any more)
        \item Explosion of droplets at supercritical pressures
    \end{itemize}
    \item Mention differences with gas turbine instabilities, unique challenges
    \begin{itemize}
        \item Combustion instabilities in gas turbine engines: operational experience, fundamental mechanisms and modeling~\cite{Lieuwen2005}
    \end{itemize}
    \item Methods of preventing/damping instabilities in rocket combustors (injector arrangement, baffles, cavities, etc.)
    \begin{itemize}
        \item A review of acoustic dampers applied to combustion chambers in aerospace industry~\cite{Zhao2015}
        \item A review of active control approaches in stabilizing combustion systems in aerospace industry~\cite{Zhao2018}
    \end{itemize}
    \item Analytical/empirical methods of predicting combustion instabilities, drawbacks
    \begin{itemize}
        \item
    \end{itemize}
\end{itemize}