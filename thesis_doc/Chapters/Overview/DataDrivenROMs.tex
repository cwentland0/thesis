\section{Data-driven Reduced-order Modeling}

In recent years, the term ``data-driven'' has become a veritable buzzword in every field imaginable, including medicine, finance, agriculture, logistics, and engineering. The term vaguely hints at advanced new technology to propel productivity and profit to new heights. Underneath the marketing hubbub, data-driven methods generally follow the same core process:

\begin{enumerate}
    \item Gather ``real-world'' information about an extremely complex system which humans cannot parse efficiently. This could be customer purchase records for an online marketplace, or biomarker responses to a new drug and the patients' health outcomes.
    \item Feed this data into an algorithm which learns to map the input data to meaningful output targets which are easily parsed by humans.
    \item Use the model to make useful predictions on new, unseen data points.
\end{enumerate}

This process, and data-driven modeling in general, are not new. Linear regression has been used extensively for over two centuries, and many statistical models still used regularly in modern times (e.g., k-means clustering, principal component analysis) were formulated over 50 years ago. What has changed in recent years is the exponential growth in computing power, data storage, and data collection mechanisms available to data scientists. As an illustrative example of this enormous technological leap, the iPhone 14 has three times the memory and approximately 1,000 times the computing speed of the Cray-2, the world's most powerful supercomputer in 1985. With this explosive increase in computational power has come a commensurate growth of modeling techniques which can make accurate predictions for highly complex non-linear problems. The many flavors of artificial neural networks have been crucial in this respect, and will be discussed later. While these technologies have enabled radical transformations of many industries through computer vision, natural language processing, and recommender systems, they have also made deep impacts in fluid flow modeling.

The phrase ``reduced-order model'' has taken on two distinct meanings in the fluid mechanics and combustion literature. The first type of ROM refers to models which are derived by simplifying complex physics with physically-meaningful (but lower-cost) approximations, or by fitting compact analytical equations to empirically-observed patterns. Examples of the former include axisymmetric models of three-dimensional systems, or flame transfer function models of gas turbines. The latter includes polynomial models of gas thermodynamic and transport properties, or Arrhenius rates for chemical reactions. For this type of ROM, the result of order reduction (relative to, perhaps, DNS or atomistic reaction models) is a secondary consequence of approximating physical processes, not of learning a lower-dimension solution to the original governing equations. In this sense, we will refer to this type of model instead as a ``reduced-physics model.'' While these models play a critical role in developing computationally-tractable models, and play a secondary role in the results presented in this thesis, they are not the focus of this work.

The second type of ROM, central to this work, refers to models which learn a mapping from a low-dimensional representation of the system state to the full-dimensional state, and provide a means of evolving this low-dimensional representation in time. At their core, these methods attempt to recreate the behavior of the full-dimensional system, not that of a physically-simplified surrogate. The low-dimensional state is often a mathematical construct without particular physical significance. In this case, order reduction is a direct consequence of the mapping from low- to high-dimensional state. As such, we believe ``reduced-order modeling'' is more appropriately applied to this type of model, and detail their historical evolution here.



\begin{itemize}
    \item Necessity of finding low-cost models for design, uncertainty quantification, failure analysis (``many-query'' buzzword)
    \item Explicit characterization of different cost-reduction methods. All are inherently ``reduced-order'', but we use that here to describe process where order reduction is the specific goal.
    \begin{itemize}
        \item Physics-inspired models
        \item Empirical surrogates
        \item Projection-based models
    \end{itemize}
    \item Early attempts at learning reaction model and combustion instability model surrogates from experiments
    \item Linear projection-based ROMs
    \item Non-linear projection-based ROMs
    \item Black/gray box
    \begin{itemize}
        \item Operator inference
        \item Machine-learning surrogates
    \end{itemize}
\end{itemize}