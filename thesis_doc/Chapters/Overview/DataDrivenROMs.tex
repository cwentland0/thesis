\section{Data-driven Reduced-order Modeling}

In recent years, the term ``data-driven'' has become a veritable buzzword in every field imaginable, including medicine, finance, agriculture, logistics, and of course engineering. The term is thrown around with abandon, hinting at advanced new technology to propel productivity and profit to new heights. Underneath the marketing hubbub, data-driven methods generally follow the same core process:

\begin{enumerate}
    \item Gather ``real-world'' information about an extremely complex system which humans cannot parse efficiently. This could be customer purchase records for an online marketplace, or biomarker responses to a new drug and the patients' health outcomes.
    \item Feed this data into an algorithm which learns to map the input data to meaningful output targets which are easily parsed by humans.
    \item Apply the model to make useful predictions on new, unseen data points.
\end{enumerate}

Data-driven modeling is not new. Linear regression has been used extensively for over two centuries, and many statistical models still used regularly in modern times (k-means clustering )

\begin{itemize}
    \item Necessity of finding low-cost models for design, uncertainty quantification, failure analysis (``many-query'' buzzword)
    \item Explicit characterization of different cost-reduction methods. All are inherently ``reduced-order'', but we use that here to describe process where order reduction is the specific goal.
    \begin{itemize}
        \item Physics-inspired models
        \item Empirical surrogates
        \item Projection-based models
    \end{itemize}
    \item Early attempts at learning reaction model and combustion instability model surrogates from experiments
    \item Linear projection-based ROMs
    \item Non-linear projection-based ROMs
    \item Black/gray box
    \begin{itemize}
        \item Operator inference
        \item Machine-learning surrogates
    \end{itemize}
\end{itemize}