\section{Organization and Notation}

The organization of this thesis is as follows. Chapter~\ref{chap:FlowModeling} details the governing equations, thermodynamic and transport models, chemical reaction models, and spatio-temporal discretization methods used to obtain results exhibited in later chapters. Chapter~\ref{chap:ProjROMs} discusses relevant literature on projection-based reduced-order modeling techniques and derives those methods which are investigated in later chapters. This includes in-depth discussions on the MP-LSVT method, scalable hyper-reduction, and adaptation of PROMS for combusting flow problems. Chapter~\ref{chap:CavityAndCVRC} exhibits numerical experiments of hyper-reduced PROMs for two intermediate-complexity multi-scale problems: 2D transonic flow over an open cavity, and a 3D truncated single-element rocket combustor. Chapter~\ref{chap:NineElement} provides results for numerical experiments of a 3D nine-element rocket combustor, the largest (and perhaps most complex) test case for PROMs to date, to the best of the author's knowledge. {\color{red}Chapter~\ref{chap:TransientFlame} outlines PERFORM, the open-source ROM development testbed, and novel results comparing linear and non-linear ROMs for an acoustically-forced, freely-propagating model premixed flame.} Finally, Chapter~\ref{chap:Conclusion} summarizes the results and meaningful conclusions of this research, and proposes several important future research directions.

A reference for the meaning of specific symbols used throughout this thesis is given in the List of Symbols. Some universal notation is used throughout. Scalars and scalar-valued functions are denoted by lowercase, italicized Latin or Greek letters (e.g., $a$, $\psi$). Vectors and vector-valued functions are denoted by lowercase, bolded Latin or Greek letters (e.g., $\mathbf{a}$, $\boldsymbol{\psi}$). Matrices and matrix-valued functions are denoted by uppercase, bolded Latin or Greek letters (e.g., $\mathbf{A}$, $\boldsymbol{\Psi}$). Vector spaces and manifolds (and sets, occasionally) are defined using uppercase, calligraphic Latin letters (e.g., $\MC{A}$, $\MC{U}$). The ``$=$'' sign denotes simple equality, while ``$\defEq$'' denotes a definition. Functions are denoted by a Latin or Greek letter, or common mathematical function abbreviation, followed by parentheses enclosing the function's arguments (e.g., $a(\cdot)$, $\mathbf{a}(\cdot)$, $\mathbf{A}(\cdot)$, $\text{exp}(\cdot)$). Brackets are used to group elements of mathematical equations (e.g., $a + [b + c]^2$), define vectors or matrices (e.g., $\mathbf{a} \defEq [a_1, \; a_2, \; a_3]$), or specify molar concentration (always written as $\moleConcSpec$). The purpose of brackets in a given context should be fairly self-evident. Braces are used to define sets (e.g., $\MC{A} \defEq \{a_1,\; a_2,\; a_3\}$). Single vertical bars indicate the absolute value operation (e.g., $|a|$), and double vertical bars indicate the vector or (induced) matrix norm (e.g., $\Vert \mathbf{a} \Vert_2$). Superscripts are generally reserved for exponentiation (e.g., $a^2$, $b^c$), indication of the time step associated with a variable (e.g., $a^{\timeIdx} \defEq a(\timeVar^{\timeIdx})$), or transpose ($\mathbf{A}^\top$), inverse ($\mathbf{A}^{-1}$), and pseudoinverse ($\mathbf{A}^+$) operations, for which the context should be self-explanatory. Subscripts are used for numerous indicators including summation or set indices (e.g., $a_{i} \; \forall \; i \in \{1,\; 2,\; 3\}$), physical value descriptors (e.g., $c_p$, $h_0$), and variable descriptors (e.g., $\consVec$ vs. $\primVec$, $\jacobResCons$ vs. $\jacobResPrim$). Diacritics are context-dependent, though we note that the macron (e.g., $\mathbf{\overline{a}}$) is generally associated with constant values, the tilde (e.g., $\mathbf{\widetilde{a}}$) is generally associated with full-dimensional ROM quantities, and the circumflex (e.g., $\mathbf{\widehat{a}}$) is generally associated with low-dimensional ROM quantities.