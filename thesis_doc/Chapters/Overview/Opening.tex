% \begin{itemize}
%     \item Computational analysis already a major component of aircraft design, particularly exterior flows over airframe and wings, and structural analysis of wings. Airframe design for liquid/solid rockets pretty simple (not so much for air-breathing).
%     \item Reacting flows (in propulsion systems) are a different beast: characteristic spatial and temporal scales are even smaller than those of associated turbulence
%     \begin{itemize}
%         \item Approximate models (linear stability, flamelet, flame transfer functions) are gross approximations that require experimental measurements, lookup tables, etc. Not generalizable.
%         \item Resolving characteristic scales requires massive computational power, well beyond the reach of anyone except academics and government researchers.
%         \item ``Accurate'' simulations have largely been restricted to canonical flows. Attempts at more practical systems can sometimes generate very different results depending on numerical models. 
%     \end{itemize}
%     \item In order for CFD to take a role in design of propulsion systems, need three things: fast, robust, and accurate. If not accurate, need some way to quantify error.
    
% \end{itemize}
